%%
%% This is file `sample-sigconf.tex',
%% generated with the docstrip utility.
%%
%% The original source files were:
%%
%% samples.dtx  (with options: `sigconf')
%% 
%% IMPORTANT NOTICE:
%% 
%% For the copyright see the source file.
%% 
%% Any modified versions of this file must be renamed
%% with new filenames distinct from sample-sigconf.tex.
%% 
%% For distribution of the original source see the terms
%% for copying and modification in the file samples.dtx.
%% 
%% This generated file may be distributed as long as the
%% original source files, as listed above, are part of the
%% same distribution. (The sources need not necessarily be
%% in the same archive or directory.)
%%
%% The ol powerpointfirst command in your LaTeX source must be the \documentclass command.
\documentclass[sigconf,screen]{acmart} %anonymous
%\documentclass[sigconf]{acmart,review}
\usepackage{xspace}

%\usepackage{amsmath,amssymb,amsfonts}
\usepackage{graphicx}
\usepackage{xcolor}
\usepackage{color, colortbl}
\usepackage[all]{nowidow}
\usepackage[inline]{enumitem}
\usepackage[scaled]{beramono}
\usepackage{booktabs} % For formal tables
\usepackage[colorinlistoftodos]{todonotes}
\usepackage[tight,footnotesize]{subfigure}
\usepackage{multirow}
\usepackage{balance}
\usepackage{listings}
%\usepackage{microtype}
\usepackage{tcolorbox}
\usepackage{cleveref}
\usepackage{soul}
\usepackage{lscape}
\usepackage{lineno}
%\usepackage[disable]{todonotes}
%%
%% \BibTeX command to typeset BibTeX logo in the docs


\AtBeginDocument{%
  \providecommand\BibTeX{{%
    \normalfont B\kern-0.5em{\scshape i\kern-0.25em b}\kern-0.8em\TeX}}}

%% Rights management information.  This information is sent to you
%% when you complete the rights form.  These commands have SAMPLE
%% values in them; it is your responsibility as an author to replace
%% the commands and values with those provided to you when you
%% complete the rights form.
%\setcopyright{acmcopyright}
%\copyrightyear{2020}
%\acmYear{2020}
%\acmDOI{10.1145/1122445.1122456}

%% These commands are for a PROCEEDINGS abstract or paper.
\acmConference[ESEM 2020]{The International Symposium on Empirical Software Engineering and Measurement}{September, 2020}{Bari, Italy}
%\acmConference[Woodstock '18]{Woodstock '18: ACM Symposium on Neural
%  Gaze Detection}{June 03--05, 2018}{Woodstock, NY}
%\acmBooktitle{Woodstock '18: ACM Symposium on Neural Gaze Detection,
%  June 03--05, 2018, Woodstock, NY}
\acmPrice{15.00}
\acmISBN{978-1-4503-XXXX-X/18/06}

\begin{document}

\title{The Name of the Title is Hope}

\title{\CT: Using collaborative filtering to recommend \GH topics}

\title{\TF: A Recommender System for Mining Relevant \GH Topics}

\title{\TF: An Approach to Recommend Relevant \GH Topics}%A Recommender 
%System to Provide Automatic \GH Topics



\author{Juri Di Rocco, Claudio Di Sipio, Davide Di Ruscio, Phuong Nguyen, Riccardo Rubei}
%	%\authornote{Dr.~Trovato insisted his name be first.}
%	\orcid{1234-5678-9012}
\affiliation{%
	\institution{Universit\`a degli studi dell'Aquila, Via Vetoio 2, 67100 -- L'Aquila, Italy}
	%\streetaddress{Via Coppito 2 -- 67100 L'Aquila, Italy}
	%	\city{Via Vetoio 2 -- 67100 L'Aquila, Italy} 
	%  \state{Ohio} 
	%  \postcode{43017-6221}
}
\email{{juri.dirocco, claudio.disipio, davide.diruscio,phuong.nguyen, riccardo.rubei}@univaq.it} % firstname.lastname




%\author{Juri Di Rocco}
%\affiliation{%
% \institution{Universit\'a degli studi dell'Aquila}
%  \streetaddress{1 Th{\o}rv{\"a}ld Circle}
%  \city{Hekla}
%  \country{Iceland}}
%}
%\email{juri.dirocco@univaq.it}



%\renewcommand{\shortauthors}{Juri Di Rocco, et al.}



% Usual suspects

\newcommand*{\ie}{i.e.,\@\xspace}
\newcommand*{\eg}{e.g.,\@\xspace}
\newcommand*{\cf}{cf.\@\xspace}
\newcommand*{\RM}{README\@\xspace}
\newcommand*{\GH}{GitHub\@\xspace}
\newcommand*{\RECOMMENDER}{<RECOMMENDER ACRONYM>\@\xspace}
\newcommand*{\CR}{<RECOMMENDER ACRONYM>\@\xspace}
\newcommand\revised[1]{\textcolor{blue}{#1}}
\newcommand{\code}[1]{{\small \texttt{#1}}}
\newcommand*{\LF}{LibFinder\@\xspace}
\newcommand*{\LC}{LibCUP\@\xspace}
\newcommand*{\LR}{LibRec\@\xspace}

\newcommand{\rqfirst}{\textbf{RQ$_1$}: \emph{How does \CR compare with the 
		state-of-the-art approaches in terms of success rate?}~} 
\newcommand{\rqsecond}{\textbf{RQ$_2$}: \emph{How well can \LR and \CR recommend third-party libraries with respect to accuracy, sales diversity, and novelty?}~}
\newcommand{\rqthird}{\textbf{RQ$_3$}: \emph{What are the reasons for the performance difference between \LR and \CR?}~}
\newcommand{\rqfourth}{\textbf{RQ$_4$}: \emph{Does an increase in the amount of input data help improve CrossRec's overall performance?}~}
%\renewcommand{\hl}{}

\newcommand*{\circled[1]}{\tikz[baseline=(char.base)]{\color{black} 
		\node[shape=circle,draw=cyan,fill=black!10!white,inner sep=.3pt] (char) {{{\texttt\textbf #1}}};}}
	
\def\checkmark{\tikz\fill[scale=0.4](0,.35) -- (.25,0) -- (1,.7) -- (.25,.15) -- cycle;} 

\makeatletter
\newcommand*{\etc}{%
	\@ifnextchar{.}%
	{etc}%
	{etc.\@\xspace}%
}
\makeatother
\newcommand*{\etal}{et~al.\@\xspace}

\newcommand{\nb}[2]{
	\fbox{\bfseries\sffamily\scriptsize#1}
	{\sf\small$\blacktriangleright$\textit{#2}$\blacktriangleleft$}
}

\newcommand\PN[1]{\textcolor{blue}{\nb{Phuong}{#1}}}
\newcommand\CDS[1]{\textcolor{red}{\nb{Claudio}{#1}}}
\newcommand\JDR[1]{\textcolor{orange}{\nb{Juri}{#1}}}

%Is the approach able to provide consistent recommendations?

\definecolor{Gray}{gray}{0.9}
 


\begin{abstract}
\textit{Background:} In the context of software development, \GH has been at the forefront of platforms to
%gained a valuable role in 
store, analyze and maintain a large number of software repositories. Topics have been introduced by \GH as an effective 
method to 
annotate stored repositories. However, labeling GitHub repositories should be 
carefully conducted to avoid adverse effects on project popularity and 
reachability.
\textit{Aims:} We present \TF, a novel approach to assist open source 
software developers in selecting suitable topics for GitHub repositories being 
created.
\textit{Method:} We built a project-topic matrix and applied a syntactic-based 
similarity 
function to recommend missing topics by representing repositories and related 
topics in a graph. %-based manner. 
The ten-fold cross-validation methodology has 
been used to assess the performance of \TF by considering different 
metrics, \ie success rate, precision, recall, and catalog coverage.
\textit{Result:} The results show that \TF recommends good topics 
depending on different factors, \ie collaborative filtering settings, 
considered datasets, and pre-processing activities. Moreover, \TF can be combined with a state-of-the-art topic recommender system (\ie MNB 
network) to improve the overall prediction performance.
\textit{Conclusion:} Our results confirm that collaborative filtering 
techniques can successfully be used to provide relevant topics for GitHub 
repositories. Moreover, \TF can gain a significant boost in prediction 
performances by employing the outcomes obtained by the MNB network as its 
initial set of topics.

%%Collaborative filtering is a well-founded technique widely used in the 
%%recommendation system domain. 
%$\blacksquare$~\textbf{Background} 
%%In recent years, a plethora of approaches have been developed to provide the 
%%users with relevant items. 
%In the context of software development, \GH has gained a precious role in 
%storing, analyzing and maintaining a considerable number of repositories.
%% In the recent years, \GH introduced the possibility to classify them 
%%employing topics to represent the stored projects in an effective manner.   
%Topics have been introduced by \GH as an effective manner to annotate stored 
%repositories. However,
%% to increase their reachability and help developers when they are 
%%searching for existing software artifacts satisfying their needs.  However,
%labeling \GH repositories should be carefully conducted to avoid 
%adverse effects on project popularity and reachability.
%%$\blacksquare$~\textbf{Aims}  In this paper, we present \CT, a recommender 
%%system to assist open source software developers in selecting suitable topics 
%%for the repositories. \CT exploits a collaborative filtering technique to 
%%recommend topics to developers by relying on the set of initial ones, which 
%%are 
%%currently included in the project being. 
%$\blacksquare$~\textbf{Aims} We present \TF, a novel approach to assist open 
%source software developers in selecting suitable topics for \GH repositories. 
%%
%%\CT exploits a collaborative filtering technique to recommend topics to 
%%developers by relying on the set of initial ones, which are currently 
%%included 
%%in the project being. 
%$\blacksquare$~\textbf{Method} We built a user-item matrix and applied a 
%syntactic-based similarity function to recommend missing topics by 
%representing 
%repositories and related topics in a graph-based manner. 
%%To the best of our knowledge, this is the first approach that uses 
%%collaborative filtering technique for recommending useful topics to a \GH 
%%repository. At which point, 
%The \emph{ten-fold cross-validation} methodology has been used to assess the 
%performance of \CT by considering different metrics, \ie precision, recall, 
%and 
%catalog coverage.
%%we validated the approach by considering different metrics, aiming to study 
%%various quality aspects (\ie which \CT parameters provide better prediction 
%%performance)., 
%$\blacksquare$~\textbf{Result} The results show how \CT recommends good topics 
%depending on different factors, \ie collaborative filtering settings, 
%considered 
%datasets and pre-processing activities. Moreover, we show how \CT  can be 
%combined with an existing topic recommender systems (\ie \MNB) to improve 
%the 
%initial overall prediction performances. 
%$\blacksquare$~\textbf{Conclusion}
%Our results confirm that collaborative filtering techniques can be 
%successfully 
%used to recommend relevant topics for \GH repositories. Moreover, \TF 
%can 
%gain a significant boost in prediction performances by employing the \MNB 
%outcomes as input topics.

\end{abstract}

%%
%% The code below is generated by the tool at http://dl.acm.org/ccs.cfm.
%% Please copy and paste the code instead of the example below.
%%

%\begin{CCSXML}
%<ccs2012>
% <concept>
%  <concept_id>10010520.10010553.10010562</concept_id>
%  <concept_desc>Computer systems organization~Embedded systems</concept_desc>
%  <concept_significance>500</concept_significance>
% </concept>
% <concept>
%  <concept_id>10010520.10010575.10010755</concept_id>
%  <concept_desc>Computer systems organization~Redundancy</concept_desc>
%  <concept_significance>300</concept_significance>
% </concept>
% <concept>
%  <concept_id>10010520.10010553.10010554</concept_id>
%  <concept_desc>Computer systems organization~Robotics</concept_desc>
%  <concept_significance>100</concept_significance>
% </concept>
% <concept>
%  <concept_id>10003033.10003083.10003095</concept_id>
%  <concept_desc>Networks~Network reliability</concept_desc>
%  <concept_significance>100</concept_significance>
% </concept>
%</ccs2012>
%\end{CCSXML}
%
%\ccsdesc[500]{Computer systems organization~Embedded systems}
%\ccsdesc[300]{Computer systems organization~Redundancy}
%\ccsdesc{Computer systems organization~Robotics}
%\ccsdesc[100]{Networks~Network reliability}

%%
%% Keywords. The author(s) should pick words that accurately describe
%% the work being presented. Separate the keywords with commas.
\keywords{Recommender systems, GitHub topics recommendation, Collaborative filtering}

\maketitle
	\section{Introduction}
\label{sec:Introduction}
In recent years, the developer community heavily exploits open source repositories during their daily activities. \GH has become one of the most
popular platforms that aggregate these projects and
support the development activity in a collaborative fashion.
%
The platform recently introduced \emph{topics} to foster the popularity and promote information discovery about popular projects. They are a set of terms used to characterize projects by summarizing their features.  
Thus, the topic labeling activity can compromise the popularity and reachability of a project if it is not properly addressed. A recent work already faced this problem by using a machine learning approach to recommend relevant topics given a README file of a repository  \cite{MNB}. However, this tool\footnote{For the sake of presentation, we refer to this work as MNB network throughout the paper} is able to recommend only \emph{featured} topics, a curated list of them provided by Github\footnote{\url{https://github.com/topics}}.

In this work, we extend the set of recommended items to non-featured topics by exploiting collaborative filtering, a widely spread technique in the recommendation system domain \cite{Schafer:2007:CFR:1768197.1768208}. Given an initial set of topics coming from a \GH project, we use repository-topic matrixes to suggest relevant topics.
The work gives the following contributions:
\begin{itemize}
\item Considering the \GH projects as products, we suggest relevant topics to the project given an initial list of them;
\item We assess the quality of the work employing a well-defined set of metrics commonly used in the recommendation system domain \ie success rate and accuracy;
\item Considering a well-founded approach, we improve it by providing an extended set of possible topics  
\end{itemize}

The rest of the work is structured as follows. Section \ref{sec:Background} shows the issues and the potential challenges in the domain. In Section \ref{sec:ProposedApproach}, we present our approach and evaluate it in Section \ref{sec:Evaluation}. We present the results of the assessment in Section \ref{sec:ExperimentalResults} and we discuss the findings. Section \ref{sec:RelatedWorks} summarizes relevant works in the field and we conclude the paper in Section \ref{sec:Conclusions} with possible future works.



\section{Motivation and background}
\label{sec:Background}
Motivation of our work should go here






\section{Proposed approach}
\label{sec:ProposedApproach}
%({\bf C}ross Project {\bf R}elationships for Computing {\bf O}pen {\bf S}ource {\bf S}oftware {\bf Sim}ilarity). 

%====================================================================================================================================
%By recommender systems,
%As described in the formal definition of the recommendation problem, at the base of each RS there are three main essential elements which are: users, items and ratings. Usually such information are represented all together by means of a user-item ratings matrix. Such ratings matrix consists of a table where each row represents a user, each column represents a specific item, and each entry represents the rating given by the user to the particular item. Usually, such matrix results very sparse in practice because users rate only a small portion of items. Fig. 2 shows an example of user-item ratings matrix in a movie RS where users express their preferences to the items (movies) by using a five points rating scale. The items with a question mark (unknown rating) are unseen for the corresponding user \cite{DBLP:conf/rweb/NoiaO15}.
%. The three main components that make up a recommender systems are 
%\paragraph{\textbf{User-item matrix}}
%\begin{figure}[h!]
%	\begin{equation} \nonumber
%	\bordermatrix{~ & she & is & today & a & nice & city \cr	
%		d_{1} 		& 1 & 1 & 0 & 0 & 1 & 0 \cr 
%		d_{2} 		& 0 & 1 & 1 & 0 & 1 & 0 \cr  
%		d_{3} 		& 0 & 1 & 0 & 1 & 2 & 1 \cr }
%	\end{equation}
%	\caption{An example of a term-document matrix}
%	\label{fig:TDM}
%\end{figure}
%; \emph{lib$_6$=commons-io:commons-io}
%====================================================================================================================================
%as an engine to generate recommendations using similarity values as its inputs.
%Given a project pi, by using CrossSim, we get a ranked list of similar projects. n is the number of most similar projects that are selected for the recommendation task. All libraries used by the top-n projects are used to compute recommendations for project pi. Number of recommended libraries: k. The number of libraries that are recommended to a test project. There are two main components in \CR: Similarity Computation and Recommendation Engine.  To evaluate a recommender system, we should take both components into account.
%====================================================================================================================================
%====================Find references that mention the importance of similarity computation and collaborative filtering. 
% they can improve productivity
%. Similar to the concept of user-item matrix.
%Afterwards, it selects the libraries using recommendation algorithms.
%With \CR we aim at supporting open source software developers by providing them with meaningful recommendations. 
%Projects with similar functionalities include similar libraries. 
%However, in the context of library recommendation, we consider the relationships of inclusion of libraries in projects.
% and an analogous user-item matrix is built to represent this relationship
%Generally, 
%====================================================================================================================================
%-library inclusion matrix of the projects, occurrence of the libraries as follows. %It is able to suggest libraries to a developer. 
% Using this representation, it is possible to compute the missing ratings by exploiting from the relationships \cite{reference/ml/MelvilleS10}. %To determine what products a given customer might like, they look for other customers who have assigned similar ratings to a similar range of products, and extrapolate from there. 
%which allows developers to select third-party libraries. 
%\eg those that help developers approach the most suitable resources.
%\subsection{Overview} \label{sec:Overview}
%It provides a library recommendation functionality, which is meaningful to OSS developers since it 
%allows them to search for third-party libraries that may be useful for their current project. In 
%this section the \CR approach is presented. 
% and third-party libraries
% on which third-party libraries should be included
%With regards to the rich metadata infrastructure available at OSS repositories, we represent the cross relationships using the graph model, so as to compute similarities among various artifacts. 
% to support OSS developers
%====================================================================================================================================

This section describes \TF, %our proposed approach to provide developers with relevant topics for \GH repositories. 
%\TF is 
a recommender system %\cite{Aggarwal2016} 
that models the relationships among OSS projects using a graph representation, 
and exploits a collaborative filtering technique 
\cite{Schafer:2007:CFR:1768197.1768208} to recommend relevant \GH topics for 
the software repository under development. Collaborative filtering techniques 
have 
been conceived in the e-commerce domain %where the relationships among products 
%and 
%users are used for 
to recommend products %predict the missing ratings of recommended items 
\cite{Linden:2003:ARI:642462.642471}, based on the assumption that \emph{``if users agree about the quality or relevance of some items, then they will likely agree about other items''} \cite{Schafer:2007:CFR:1768197.1768208}. 
%Similarly to the work presented in~\cite{NGUYEN2020110460}, our proposed technique follows the assumption that \emph{``if users agree about the quality or relevance of some items, then they will likely agree about other items''} \cite{Schafer:2007:CFR:1768197.1768208}. 
%In a similar fashion, 
\TF works %recommends topics %works in the same manner to %aims to solve the problem of the reachability of a \GH repository given a set of topics. In particular, we recommend a set of topics 
following the same line of reasoning to mine \GH topics: \emph{``if projects have some tags in common, then they will probably contain other relevant tags''}~\cite{NGUYEN2020110460}.

\begin{figure}[t!]
	\centering
	\includegraphics[width=\linewidth]{figs/TopFilter.pdf}
	\caption{Overview of the \TF architecture.}% [??Rename Similarity 
	%Computation with Similarity Calculator
	\label{fig:TopFilterArchitecture}
\end{figure}


%This is a preparatory phase for the next steps of the recommendation process. The developer interacts with \TF by sending a request for recommendations that includes the list of topics which she has already included for the repository that she is working on.
%The aforementioned components are singularly described in the next sections. 

In the following subsections, we describe two \TF configurations %, namely \TFa and \TFb 
to recommend relevant topics by incorporating different types of input 
data. The first configuration exploits a collaborative-filtering technique, while the second one is a 
combination of both \MNB and \TFa, aiming to improve the prediction performance 
of the original \MNB approach as proposed by the authors in their paper \cite{10.1145/3383219.3383227}. % by combining it with \TFa. %\TF can be 
%%combined with \MNB to improve. %the collaborative filtering prediction 
%%%performance.

  %in detail each of the components mentioned above. %In Section~\ref{sec:combined_approach}, 
%Furthermore, based on the proposed architecture, we present an approach to improve \MNB by 

 % shown in Fig. \ref{fig:CrossRecArchitecture} \cite{CROSSREC-DATA}
%\vspace{-.2cm}
%analyzes the input to 
%CrossRec represents the data collected from OSS repositories.
%. The data collected from OSS repositories is converted into a matrix
%A matrix of this type is very sparse since most users give ratings to a small amount of items. 


%\subsection{Recommender}



\subsection{\TFa: Recommending topics using a collaborative-filtering technique} \label{sec:CFRecommendation}

%First \TF represents the relationships among projects\footnote{For the sake of presentation, the terms ``repositories'' and ``projects'' are used interchangeably throughout the paper} and topics retrieved from existing repositories as a graph. %Next, the graph representation are involved to compute similar projects to that under development. Finally, relevant topics are recommended for the input project using a collaborative-filtering technique.

%This section presents the framework to. 

Figure~\ref{fig:TopFilterArchitecture} depicts the architecture conceived to realize the \TF prototype. First, %the \code{Topic Cleaner} component is used to filter out very rare topics according to their frequencies of occurrence over all projects in the initial dataset. Afterwards, 
the \code{Data Encoder} component encodes \GH repositories in a graph-based 
representation, and then \code{Similarity Calculator} computes similarities 
among all the managed projects. The \code{Recomme\-ndation Engine} component 
implements a \emph{collabora\-tive-filtering} 
technique
%~\cite{Aggarwal2016,Zhao:2010:UCR:1748610.1749278,NGUYEN2020110460} 
%,\ it selects top-$k$ similar repository, and performs computation 
to generate a ranked list of \emph{top-N} topics which %.~Finally, a ranked list of topics 
is eventually suggested to the developer to complement an initial list of 
topics given as input. We explain in detail the functionalities of each 
component as follows.


\subsubsection{Data Encoder} \label{sec:DataEncoder}

%Considering traditional recommender systems for online services, we can identify three main components, namely \emph{users}, \emph{items}, and \emph{ratings} \cite{Sarwar:2001:ICF:371920.372071},\cite{DBLP:conf/rweb/NoiaO15}. 

The mutual relationships among \GH repositories and topics is encoding using a 
\emph{project-topic matrix}~\cite{Sarwar:2001:ICF:371920.372071}: Each row 
represents a project, and each column corresponds to a topic. In this sense, a 
cell in the matrix is set to 1 if the project in the corresponding row is 
tagged with the topic in the corresponding column, otherwise the cell is set to 
0.

To build the project-topic matrix, raw topics are first pre-process\-ed using various Natural Language Processing (NLP) techniques, such as stemming, lemmatization, and stop words removal. This aims to remove possible syntactical duplicates terms, \eg \textit{document} and \textit{documents}, which are frequent in \GH. Afterwards, the final matrix is constructed by means of the topics obtained through the pre-processing phase. %More specifically, stemming, lemmatization, and stop words are applied on the mined topics.

%the items are represented as columns, and each cell in the matrix corresponds to a rating given by a user for an item \cite{DBLP:conf/rweb/NoiaO15}. Translating this matrix in our domain, users are substitute by projects as well as topics are the possible items to recommend. The resulting \emph{repo-topic ratings matrix} represents possible relationships between these two elements \ie project may include various topics. 

%Pre-processing tasks are performed to transform a metamodel into a feature vector [38], i.e., X = (x1, x2, ..., xL), L is the number of input neurons (see Fig. 2). We illustrate how the Data Extractor 2 works by means of Fig. 4. The Term Extractor E parses terms from the input metamodel. Then, the extracted raw terms are normalized by the NLP Normalizator N that performs the following Natural Language Processing (NLP) steps: (i) stemming (ii) lemmatization, and (iii) stop words removal [23], [55].

%We can denote \emph{project-topic inclusion} relationships as $\ni$. In this matrix, each row represents a project and each column represents a topic. A cell in the matrix is set to $1$ if the topic in the column is included in the project specified by the row, it is set to $0$ otherwise. For the sake of clarity and conformance, we still denote this as a user-item ratings matrix throughout this paper.

For explanatory purposes, we consider a set of four projects 
$P=\{p_1,p_2,p_3,p_4 \}$ together with a set of topics $L=\{$\emph{t$_1$} = 
\emph{machine-learning}; \emph{t$_2$} = \emph{javascript}; \emph{t$_3$} = 
\emph{database}; \emph{t$_4$} = \emph{web}; \emph{t$_5$} = \emph{algorithm}, 
\emph{t$_6$} = \emph{algorithms}$\}$. Moreover, the \emph{project-topic 
inclusion} relationships is denoted as $\ni$. By parsing the projects, we 
discover the following inclusions: $p_1$ $\ni$ $t_1,t_2, t_6$; $p_2 \ni 
t_1,t_3$; $p_3 \ni t_1 ,t_3, t_4, t_5$; $p_4 \ni t_1,t_2,t_4,t_5$. After the 
NLP normalization steps, the topics \emph{t$_5$} and \emph{t$_6$} collapse on 
the same term which is named as \emph{t$_{5}$}. The final project-topic matrix 
is shown in Table~\ref{tab:repo-topic-matrix}. 
%and the corresponding graph-based 
%representation is shown in Fig. \ref{fig:Graph}. %is generated to represent 
%inclusions between repositories and topics.

\begin{table}[h!]
	\caption{The \emph{project-topic matrix} of the explanatory example.}
	\begin{tabular}{|p{0.5cm}|p{0.5cm}|p{0.5cm}|p{0.5cm}|p{0.5cm}|p{0.5cm}|p{0.5cm}|} \hline
		& \emph{t$_1$} & \emph{t$_2$} & \emph{t$_3$} & \emph{t$_4$} & \emph{t$_{5}$} \\ \hline
		$p_1$ & 1 & 1 & 0 & 0 & 1 \\ \hline
		$p_2$ & 1 & 0 & 1 & 0 & 0 \\ \hline
		$p_3$ & 1 & 0 & 1 & 1 & 1 \\ \hline
		$p_4$ & 1 & 1 & 0 & 0 & 1 \\ \hline
	\end{tabular}
    \label{tab:repo-topic-matrix}
\end{table}




% Accordingly, the user-item ratings matrix built to model the occurrence of the topic is depicted in Fig.~\ref{fig:UserItemMatrix}.
%====================================================================================================================================
%\footnote{The file \emph{pom.xml} defines all project dependencies with external Maven libraries (\url{https://maven.apache.org/guides/introduction/introduction-to-the-pom.html}) }
%By performing an observation on a data set consisting of more than one thousand GitHub Java projects, we found out that the project-library inclusion matrix is very sparse since most projects contain a limited number of libraries, whereas the total number of libraries included by the projects is pretty large.

%There are two key components: Similarity Computation and Recommendation Engine. The Similarity computation module performs. Based on the list of libraries. Similarity computation is important.
%The former is used to compute similarity between different artifacts, \eg projects, libraries, or even developers. 
%====================================================================================================================================
%is of highly importance. The ability to find most similar projects 
%Similarity computation plays a key role in. Measuring the similarities between developers and software projects is a critical phase for most types of recommender systems \cite{DBLP:conf/rweb/NoiaO15}. Similarities are used as a base by both content-based and collaborative-filtering recommender systems to choose the most suitable and meaningful items for a given item \cite{Schafer:2007:CFR:1768197.1768208}. Failing to compute precise similarities means concurrently adding a decline in the overall performance of these systems. 
%====================================================================================================================================
%Using a recommendation algorithm, CrossRec selects the top most libraries as recommendations. The Recommendation Engine searches for libraries that are included in the similar projects. 
%We also investigate the effect of similarity computation by considering two similarity metrics. 
%Currently, CrossRec supports only GitHub, for future implementation, we expect to include various.
%To incorporate. Based on two premises: Projects use similar libraries to implement similar functionalities. And similar projects use similar set of libraries. To this end, we concentrate on finding a practical solution to represent the relationships between the artifacts, and eventually to compute similarity and cluster OSS projects.
%To compute the similarity between two open software projects. The graph representation allows for different similarity algorithms. In the scope of this paper, we choose the algorithm proposed by \emph{Di Noia et al.} for computing. For future work, other algorithms can also be flexibly included as long as they are suitable for graph.
%: either (i) there are direct links between them; or (ii)
%; or (iii) they are pointed by the same subject with the same property
%For library recommendation, we consider only one property, i.e. \emph{includes} (See Fig. \ref{fig:Graph}). 
%The similarity between $\alpha$ and $\beta$ 
%on a set of properties $P$ is the weighted mean of the values by all properties in $P$ as given below:  
%\begin{equation*}
%VsmSim(\alpha,\beta)=\frac{\sum_{p\in P} \omega_p VsmSim_{p}(\alpha,\beta)}{|P|}
%\end{equation*}
%to compute similarity and finally to provide inputs for a recommender system. 
%====================================================================================================================================
% makes, similarity computation becomes more complicated as many artifacts and several cross relationships prevail.
%As the input for the recommendation process, it is necessary to compute the similarity among software projects. 
%The ability of recommendation is important in the context of mining software repositories. 
%Understanding the similarity between software projects allows for reusing of codes and prototyping, or choosing alternative implementations \cite{10.1109/SANER.2017.7884605}. 
%We see that the ability to measure the similarity between artifacts, \eg projects, code snippets, or even developers, is of highly importance. Meanwhile measuring the similarity between developers and software projects is a critical phase for most types of recommender systems. 
%By considering the analogy of typical applications of RDF graphs and the problem of detecting the similarity of open source projects, in this section we propose. 
%to different open source software projects and 
%an approach that makes use of graphs for representing different kinds of relationships in the OSS ecosystem. 
%One of its applications is to compute similarity computation for supporting recommender systems \cite{DiNoia:2012:LOD:2362499.2362501}.
%With the adoption of the graph representation, it is possible to compute similarity among artifacts exploiting numerous number of graph similarity algorithms. 
%Based on the graph structure, one can exploit nodes, links and the mutual relationships to compute similarity using existing graph similarity algorithms. 
% are incorporated into the similarity calculation
% , given an input project \emph{p}, CrossRec searches for libraries by considering the most similar projects to \emph{p}. 
%====================================================================================================================================
%The rich metadata infrastructure available from the \projectName Knowledge Base is attributed to various artifacts, such as source code, API calls, forum discussions, and bug reports. 
%To this end, we concentrate on finding a practical solution to represent the relationships between the artifacts, and eventually to compute similarity and cluster OSS projects.
%We believe that a homogeneous and formal representation of the intrinsic features of OSS repositories is needed to effectively support project similarity computation.
%In this system, either humans or non-human factors have mutual dependency and implication on the others. 
%A directed graph is defined as a tuple $G=(V,E,R)$, with $V$ being the set of vertices, $E$ being the set of edges and $R$ representing the relationship among the nodes. 
% and Fig. \ref{fig:UserItemMatrix}. 
%sketches the graph representation .  the representation for 
%Specifically, the graph model has been chosen since it allows 
%\todo[size=\tiny, color=green!40]{Nr. 11}
%====================================================================================================================================




\subsubsection{Similarity Calculator} \label{sec:SimilarityCalculator}

%The \code{Recommendation Engine} of \TF works by relying on an analogous user-item ratings matrix. To provide 
%inputs for this module, the first task of \TF is to perform similarity computation on its input data to find the 
%most similar projects to a given project. In this respect, the ability to compute the similarities among projects has 
%an effect on the recommendation outcomes.~Nonetheless, computing similarities among software systems is considered to 
%be a difficult task \cite{McMillan:2012:DSS:2337223.2337267}. In addition, the diversity of artifacts in OSS 
%repositories as well as their cross relationship makes the similarity computation become even more complicated. In OSS 
%repositories, both humans (\ie developers and users) and non-human actors (such as repositories, and libraries) have 
%mutual dependency and implication on the others. The interactions among these components, such as developers commit to 
%or star repositories, or projects include libraries, create a tie between them and should be included in similarity 
%computation.
%
%%Furthermore, the graph structure also facilitates graph kernel methods, which are an effective way to compute similarities \cite{ODMD14a}. 
%%As being inspired by the research from the related Linked Data and Semantic Web field \cite{bizer_linked_2009}, 
%% as done \eg in ~\cite{Nguyen:2015:CRV:2942298.2942305,NDRDSEAA2018}
%% \hl{The representation is inspired by the one presented in}~\cite{NDRDSEAA2018}, however the relationship between projects and libraries is the inverse semantic path, i.e. \texttt{includes} instead of \texttt{isUsedBy}.
%
%
%We assume that a representation model that addresses the semantic 
%relationships among miscellaneous factors in the OSS community is beneficial to 
%project similarity computation. To this end, we consider the community of developers together with 
%OSS projects, topics, and their mutual interactions as an 
%\textit{ecosystem}. We derive a 
%\textit{graph-based} model to represent different kinds of relationships in the OSS ecosystem, and 
%eventually to calculate similarities. In the context of mining OSS repositories, the graph model is a convenient approach since it allows for flexible data integration and numerous computation techniques. 
%By applying this representation, we are able to transform the set of projects and topics shown in Fig.~\ref{fig:UserItemMatrix} into a directed graph as in Fig.~\ref{fig:Graph}.
%%\todo[size=\tiny, color=green!40]{This refers to our SEAA paper} 
%%In Linked Data, an RDF\footnote{RDF 1.1 Concepts and Abstract Syntax: \url{https://www.w3.org/TR/2014/REC-rdf11-concepts-20140225/}} graph is made up of an enormous number of nodes and oriented links with semantic relationships. Thanks to this feature, the similarity of two nodes in a graph can be computed by considering their intrinsic characteristics like neighbour nodes and their mutual interactions \cite{DiNoia:2012:LOD:2362499.2362501,Nguyen:2015:CRV:2942298.2942305}.
%~We adopted our proposed CrossSim approach~\cite{Nguyen:2019:FRS:3339505.3339636},\cite{8498236} to compute the 
%similarities among OSS graph nodes. It relies on techniques successfully 
%exploited by many studies to do the same task 
%\cite{DiNoia:2012:LOD:2362499.2362501},\cite{BRIGUEZ20146467}. Among 
%other relationships, two nodes are deemed to be similar if they point to the same node with the same edge. By looking at the graph in 
%Fig.~\ref{fig:Graph}, we can notice that $p_3$ and $p_4$ are highly 
%similar since they both point to three nodes $t_{1}, t_{4}, t_{5}$.  This 
%reflects what also suggested in a previous work by McMillan \etal 
%\cite{McMillan:2012:DSS:2337223.2337267}, \ie similar projects implement common 
%pieces of functionality by using a shared set of libraries.
%
%
%\begin{figure}[t!]
%	\centering
%	\includegraphics[width=\columnwidth]{figs/Graph.pdf}
%	\caption{Graph representation for projects and libraries.}
%	\label{fig:Graph}
%\end{figure}


%Many recommender systems rely heavily on similarity metrics to suggest suitable and meaningful recommendation for a given  input~\cite{Schafer:2007:CFR:1768197.1768208,Sarwar:2001:ICF:371920.372071,NGUYEN2020110460}. 
%Similarly, 
%The \code{Recommendation Engine} 



%\emph{Similarity Calculator} applies a similarity function on the mentioned project-topic matrix to provide very closest project in a given initial set to the input one. 

%As a result, computing properly this similarity score affects the quality of recommendation outcomes. Nonetheless, computing similarities among topics could be a daunting task. \GH allows any repository owner to add, change, or delete the list of topics that describe his project. This impacts on the stability of the topics, as they can change rapidly over time. In addition, a developer can freely specify the entire set of topics. This makes the similarity computation more complicated, as some topics couldn't have a semantic link with the others. Moreover, we can miss some key relationships depending on the similarity function employed by the calculator. For example, a purely syntactic-based similarity function assign a lower score to the topic pair 3d-graphics even though these two terms are strongly bounded in their meaning.

%a profitable and flexible data representation for enabling numerous computation techniques such as the one defined ~\cite{Nguyen:2019:FRS:3339505.3339636},\cite{8498236} 

This component relies on the previously encoded data to assess the similarity 
of given repositories. For explanatory reasons, we represent a set of projects 
and their topics in a graph, so as to calculate the similarities among the 
projects.For instance, Figure~\ref{fig:Graph} depicts the graph-based 
representation of the project-topic matrix in Table~\ref{tab:repo-topic-matrix}.  
%
Two nodes in a graph are considered to be similar if they share the same neighbors by considering their edges. Such a technique has been successfully exploited by many studies to do the same task~\cite{BRIGUEZ20146467} in different domains.

%,\cite{DiNoia:2012:LOD:2362499.2362501}
%By looking at the graph in Fig.~\ref{fig:Graph}, we notice that $p_1$ and $p_2$ share two neighbor nodes, \ie  $t_{2}$ and $t_{3}$. %From the graph, we can also learn additional information about the topics themselves. For example, $t_{3}$ seems a very popular term since is pointed by three different projects. 
%The similarities calculator consider two project nodes $p$ and $q$ as similar in an OSS graph by considering their feature sets. 

Given that $p$ has a set of neighbor nodes ($t_{1}$, $t_{2}$,.., 
$t_{l}$), the features of $p$ are represented by a vector 
$\phi=(\phi_{1},\phi_{2},..,\phi_{l})$, with $\phi_{i}$ being the weight of 
node $t_{i}$ computed as the \emph{term-frequency inverse document 
frequency} function as follows: $\phi_{i} = f_{t_{i}} \times log( \left | P 
\right | \times a_{t_{i}}^{-1} )$, where $f_{t_{i}}$ is the number of 
occurrences of $t_{i}$ with respect to $p$, it can be either $0$ and $1$. 
%since there is a maximum of one $t_{i}$ connected to $p$ by the 
%corresponding inclusion relation; 
$\left | P \right |$ is the total number of 
considered 
projects; $a_{t_{i}}$ is the number of projects connecting to $t_{i}$ 
via corresponding edges.
% \emph{includes}.
%\texttt{tf-idf}   of the node 
%\vspace{-.1cm}
%\begin{equation}\label{eqn:TFIDF}
%\phi_{i} = f_{t_{i}} \times log(\frac{ \left | P \right |}{a_{t_{i}}})
%\end{equation}
%In the meanwhile, $t_1$ and $t_4$ are used only by one project, $p_1$ and $p_3$ respectively. 
%This reflects what also suggested in a previous work by McMillan \etal
%\cite{McMillan:2012:DSS:2337223.2337267}, \ie similar projects implement common
%pieces of functionality by using a shared set of libraries.
\begin{figure}[t!]
\centering
\includegraphics[width=0.8\columnwidth]{figs/graphCFtop.pdf}
\caption{Graph representation for projects and topics.}
\label{fig:Graph}
\end{figure}
%\noindent
%According to Eq.~\ref{eqn:TFIDF}, node $t_{1}$ in Fig.~\ref{fig:Graph} has a low weight compared to that of other nodes since it is pointed by all four project nodes. In practice, this is comprehensible since \emph{junit:junit} is a very popular dependency and thus it should have a less important role in characterizing a project.
Intuitively, the similarity between two projects $p$ and $q$ with their corresponding feature vectors $\phi=\{\phi_{i}\}_{i=1,..,l}$ and $\omega=\{\omega_{j}\}_{j=1,..,m}$ is computed as the cosine of the angle between the two vectors as given below. %Eventually, the similarity between $p$ and $q$ with their corresponding feature vectors $\phi=\{\phi_{i}\}_{i=1,..,l}$ and $\omega=\{\omega_{j}\}_{j=1,..,m}$ is computed using the cosine similarity function:%\todo[size=\tiny, color=green!40]{Nr. 3}
%\vspace{-.2cm}
% $p$ and $q$ are characterized by using vectors in an $n$-dimensional space, and Eq.~\ref{eqn:VsmSim} measures
\begin{equation} \label{eqn:VsmSim}
sim(p,q)=\frac{\sum_{t=1}^{n}\phi_{t}\times \omega_{t}}{\sqrt{\sum_{t=1}^{n}(\phi_{t})^{2} }\times \sqrt{\sum_{t=1}^{n}(\omega_{t})^{2}}} 
\end{equation}

%\noindent
where $n$ is the cardinality of the set of topics that $p$ and $q$ share in common.% \cite{DiNoia:2012:LOD:2362499.2362501}. %and it is computed using the inner product as follows.







%\textbf{??Intuitive description of eq 2 is needed.}
%====================================================================================================================================
%in which $\omega_p$ is the weight for property $p$ and computed using a genetic algorithm. 
%The hypothesis is based on the fact that the projects are aiming at creating common functionalities by using common libraries.
%In \cite{Jeh:2002:SMS:775047.775126}, SimRank has been developed to calculate similarities based on mutual relationships between graph nodes. Considering two nodes, the more similar nodes point to them, the more similar the two nodes are. In this sense, the similarity between two nodes $\alpha,\beta \in V$ is computed by using a fixed-point function. Given $k \geq 0$ we have $R^{(k)}(\alpha,\beta) = 1$ with $\alpha = \beta$ and $R^{(k)}(\alpha,\beta) = 0$ with $k=0$ and $\alpha \neq \beta$, SimRank is computed as follows:
%
%\begin{equation}\label{eqn:SimRank}
%R^{(k+1)}(\alpha,\beta) = 
%\frac{\Delta}{|I(\alpha)|\cdot|I(\beta)|}\sum_{i=1}^{|I(\alpha)|}\sum_{j=1}^{|I(\beta)|}R^{(k)}(I_{i}(\alpha),I_{j}(\beta))
%\end{equation}
%
%where $\Delta$ is a damping factor ($0 \leq \Delta < 1$); $I(\alpha)$ and $I(\beta)$ are the set of incoming neighbors of $\alpha$ and $\beta$, respectively. $|I(\alpha)|\cdot|I(\beta)|$ is the factor used to normalize the sum, thus forcing $R^{(k)}(\alpha,\beta) \in [0,1]$. 
%
%For the first implementation of CrossSim we adopt SimRank as the mechanism for computing similarities among OSS graph nodes. For future work, other similarity algorithms can also be flexibly integrated into CrossSim, as long as they are designed for graph. %as long as. is able to incorporate various similarity algorithms, 
%\begin{figure}[t!]
%	\centering
%	\includegraphics[width=0.3\textwidth]{figs/SimRank.pdf}
%	\caption{SimRank similarity}
%	\label{fig:SimRank}
%\end{figure}
%====================================================================================================================================
%Equation~\ref{eqn:SimRank} implies that the similarity for two nodes is computed by aggregating the similarity of all possible pairs of their neighbors. 
%We are convinced that the utilization of SimRank is convenient and practical also when various relationships are incorporated into the graph. Given the circumstances, the algorithm needs not be changed since it only works on the basis of nodes and edges. In this sense, 
% \emph{CrossSim} is a versatile similarity tool as it can accept various input features regardless of their format. 
%To study the performance of CrossSim we conducted a comprehensive evaluation using a real dataset collected from GitHub. To aim for an unbiased comparison, we opted for existing evaluation methodologies from other studies of the same type \cite{Lo:2012:DSA:2473496.2473616,McMillan:2012:DSS:2337223.2337267,10.1109/SANER.2017.7884605}. Together with other metrics typically used for evaluations, i.e. \textit{Success rate}, \textit{Confidence},  \textit{Number of false positives}, and \textit{Precision}, we decided to use also \textit{Ranking} to measure the sensitivity of the similarity tools to ranking results. The details of our evaluation are given in the next section. In the first place, it is necessary to compute similarities among projects.
%In this case, the ratings provided by similar users to a target user A are used to make recommendations for A. %The predicted ratings of A are computed as the weighted average values of these “peer group” ratings for each item. 
%From: https://www.sciencedaily.com/releases/2017/12/171206122420.htm
%The recommendation systems at websites such as Amazon and Netflix use a technique called "collaborative filtering."
%To compute the missing ratings, generally there are two ways corresponding to the way we exploit the user-item matrix. In this paper, we investigate both types of collaborative-filtering recommender system: user-based and item-based.
%\paragraph{\textbf{User-based collaborative filtering}}
%One of the most promising such technologies is collaborative filtering [19, 27, 14, 16]. 
%works by filtering or evaluating items using the preferences of other users. 
%In this approach personalized recommendations for a target user are generated using opinions of users having similar tastes to those of the target user. The main assumption in this approach is that users with similar preferences in the past will have similar preferences in the future.
% To compute the missing ratings, there are two main ways, which are basically based on the column-wise and row-wise relationships of the user-item ratings matrix. For \emph{user-based collaborative filtering}. Exploit the relationships among users to deduce the missing ratings. 
%The missing ratings for a given user are computed by exploiting the existing ratings from other similar users. Row-wise computation, among the top most similar projects. 
%====================================================================================================================================
%By online systems, collaborative filtering works by searching for similar. by building a database of preferences for items by users. A new user, Neo, is matched against the database to discover neighbors, which are other users who have historically had similar taste to Neo. Items that the neighbors like are then recommended to Neo, as he will probably also like them. Collaborative filtering has been very successful in both research and practice, and in both information filtering applications and E-commerce applications. However, there remain important research questions in overcoming two fundamental challenges for collaborative filtering recommender systems (Item-based Collaborative Filtering Recommendation Algorithms).
%In addition, collaborative-filtering is considered to be better. 
%from these. calculates similarity between users by comparing their ratings on the same item, and it then computes the predicted rating for an item by the active user
%similar to the active user where weights are the similarities of these users with the target item
%(i.e. the user requiring a prediction) 
%====================================================================================================================================

%A collaborative-filtering recommender system suggests products that customers similar to the customer being considered have already purchased. 

\subsubsection{Recommendation Engine}

%In the context of \TF, we exploit the user-based collaborative-filtering technique for implementing the recommendation engine. 

%Consider the mutual relationships between a project and its topics represented in a graph data structure, we exploit the user-based collaborative-filtering technique to implement \TF's recommendation engine. In this context, we adapted the concept of \emph{rating} to describe the appearance of a specific topic in a project and the employed collaborative filtering techniques aim to find additional similar topics. Moreover, we name the project that needs prediction for topic suggestion as \emph{active project}. Figure~\ref{fig:UserBasedCF} depicts an instance of repo-topic rating matrix where the row $p$ is the active project and an asterisk ($*$) represents a known rating (\ie $1$ if the topics is already included, $0$ otherwise), whereas a question mark ($?$) represents an unknown rating and needs to be predicted.


%\begin{figure}[t!]
%\centering
%%	\vspace{-.4cm}	
%\includegraphics[width=0.55\linewidth]{figs/matrix.pdf}
%\vspace{-.4cm}
%\caption{Deduction of missing ratings~\cite{Zhao:2010:UCR:1748610.1749278}.}% [?? In the previous version there was a rectangle including some of the rows. Isn't?]
%\vspace{-.1cm}
%\label{fig:UserBasedCF}
%\end{figure}

% from the ratings matrix
%using the user-based collaborative-filtering technique



Given an input project $p$, and an initial set of related topics decided by the 
developer, the inclusion 
of 
additional topics can be predicted from the projects that are similar to $p$. 
In other words, \TF predicts topics' presence by means of those collected from 
the \emph{top-k} similar projects using the following 
formula~\cite{NGUYEN2020110460}:

%Referring to the example depicted in Figure~\ref{fig:UserBasedCF}, the relationships between the active project $p$ and the similar projects $q_1,q_2,q_3$ are used to compute the missing topic ratings for $p$. 
%The rectangles in Fig. \ref{fig:UserBasedCF} imply that the row-wise relationships between the active project $p$ and the similar projects $q_1,q_2,q_3$ are exploited to compute the missing ratings for $p$.
%The presence of a topic $t$ in project $p$ is deduced by means of the following formula:
%The following formula is used to predict if $p$ should include a topic $t$, \ie~$p \ni t$:%~\cite{DBLP:conf/rweb/NoiaO15}: % the predicted value $r_{p,l}$ is computed
%\vspace{-.2cm}
\begin{equation} \label{eqn:Prediction}
r_{p,t}=\overline{r_{p}}+\frac{\sum_{q \in topsim(p)}(r_{q,t}-\overline{r_{q}})\cdot sim(p,q) }{\sum_{q \in topsim(p)} sim(p,q) } %\left | P \right |
\end{equation}

\noindent
where $\overline{r_{p}}$ and $\overline{r_{q}}$ are the mean of the ratings of $p$ and $q$, respectively; $q$ belongs to the set of \emph{top-k} most similar projects to $p$, denoted as $topsim(p)$; $sim(p,q)$ is the similarity between the active project and a similar project $q$, and it is computed using Equation \ref{eqn:VsmSim}. %For a testing project $p$, $\overline{r_{p}}$ is equal to $1$ since the ratings for all testing libraries are $1$.





%\subsection{Recommendation Engine} \label{sec:RecommendationEngine}
%\PN{Please reprhare this section}
%The representation using a user-item ratings matrix allows for the computation of missing ratings \cite{Aggarwal2016},\cite{DBLP:conf/rweb/NoiaO15}. Depending on the availability of data, there are two main ways to compute the unknown ratings, namely \emph{content-based} \cite{Pazzani2007} and \emph{collaborative-filtering} \cite{Miranda:2008:ICF:1486927.1487083} recommendation techniques. The former exploits the relationships among items to predict the most similar items. The latter computes the ratings by taking into account the set of items rated by similar customers. There are two main types of collaborative-filtering recommendation: \emph{user-based} \cite{Zhao:2010:UCR:1748610.1749278} and \emph{item-based} \cite{Sarwar:2001:ICF:371920.372071} techniques. As their names suggest, the user-based technique computes missing ratings by considering the ratings collected from similar users. Instead, the item-based technique performs the same task by using the similarities among items \cite{Cremonesi:2008:EMC:1468165.1468327}.
%
%In the context of \CR, the term \emph{rating} is understood as the occurrence of a library in a project and computing missing ratings means to predict the inclusion of additional libraries. The project that needs prediction for library inclusion is called the \emph{active project}. By the matrix in Fig. \ref{fig:UserBasedCF}, $p$ is the active project and an asterisk ($*$) represents a known rating, either $0$ or $1$, whereas a question mark ($?$) represents an unknown rating and needs to be predicted.
%
%%Because of the above assumptions, the collaborative filtering algorithm is based on the comparison of one user’s behavior with other user’s behavior, to find his nearest neighbors, and according to his neighbor’s interests or preferences to predict his interests or preferences.
%
%\begin{figure}[t!]
%	\centering
%	%	\vspace{-.4cm}
%	\includegraphics[width=0.7\linewidth]{figs/UserBasedCF.pdf}
%	\vspace{-.4cm}
%	\caption{Computation of missing ratings using the user-based collaborative-filtering technique~\cite{Zhao:2010:UCR:1748610.1749278}.}% [?? In the previous version there was a rectangle including some of the rows. Isn't?]
%	\vspace{-.1cm}
%	\label{fig:UserBasedCF}
%\end{figure}
%
%%\CR employs a collaborative-filtering technique   based on the similar model applied in e-commerce systems
%
%Given the availability of the cross-relationships as well as the possibility to compute similarities among projects using the graph representation, we exploit the user-based collaborative-filtering technique as the engine for recommendation \cite{Linden:2003:ARI:642462.642471,Zhao:2010:UCR:1748610.1749278}. Given an active project $p$, the inclusion of libraries in $p$ can be deduced from projects that are similar to $p$. The process is summarized as follows: % \cite{Cacheda:2011:CCF:1921591.1921593}:
%%In particular, the user-based collaborative-filtering technique predicts a missing rating by considering the most similar projects to $p$. 
%
%%%The engine first searches for similar projects and then computes missing ratings as a weighted average of the ratings of the items by projects .
%%According to [16, 15], item-based CF algorithms provide better performance and quality than user-based ones. Afterwards the unknown ratings are computed by considering the libraries included in these projects. In this paper, the user-based technique is utilized. 
%
%\begin{itemize} 	
%	\item Compute the similarities between the active project and all projects in the collection;
%	\item Select \emph{top-k} most similar projects; and %a subset of the users (neighborhood) according to their similarity with the active user;
%	\item Predict ratings by means of those collected from the most similar projects.
%\end{itemize} 
%
%The rectangles in Fig. \ref{fig:UserBasedCF} imply that the row-wise relationships between the active project $p$ and the similar projects $q_1,q_2,q_3$ are exploited to compute the missing ratings for $p$. The following formula is used to predict if $p$ should include $l$, \ie~$p \ni l$~\cite{DBLP:conf/rweb/NoiaO15}: % the predicted value $r_{p,l}$ is computed
%%\vspace{-.2cm}
%\begin{equation} \label{eqn:Prediction}
%r_{p,l}=\overline{r_{p}}+\frac{\sum_{q \in topsim(p)}(r_{q,l}-\overline{r_{q}})\cdot sim(p,q) }{\sum_{q \in topsim(p)} sim(p,q) }  %\left |  P \right |  
%\end{equation}
%
%\noindent
%where $\overline{r_{p}}$ and $\overline{r_{q}}$ are the mean of the ratings of $p$ and $q$, respectively; $q$ belongs to the set of \emph{top-k} most similar projects to $p$, denoted as $topsim(p)$; $sim(p,q)$ is the similarity between the active project and a similar project $q$, and it is computed using Equation \ref{eqn:VsmSim}. %For a testing project $p$, $\overline{r_{p}}$ is equal to $1$ since the ratings for all testing libraries are $1$. 


%\subsection{CrossRec in Action}
%\label{sec:inaction}
%In the context of the EU H2020 CROSSMINER project\footnote{\url{https://www.crossminer.org}}, we integrated \CR 
%into Eclipse as shown in Fig.~\ref{fig:IDE}. 
%The figure describes a development of an explanatory scenario where a 
%	developer is improving a software project, named \textit{aethereal} hereafter, 
%	by replacing some existing source code with features provided by third-party 
%	libraries. The goal of such changes is to make the code easier to be understood 
%	and evolved. The project is a command line tool written in Java that aims at 
%	supporting the automatic generation of cross-projects migration dependencies 
%	graph. To this aim \textit{aethereal} distinguishes between clients and 
%	libraries: both clients and libraries are Maven artifacts, and clients are 
%	defined as artifacts that use a specific  library available on the Maven 
%	repository.
%
%We explain how \CR helps the developer evolve the  \code{build} method of the class \code{MavenDataset} as follows.
%	Such a  method computes a dependency matrix between client versions and libraries. The initial implementation of the \code{build} method prints the logging information to the console by using Java I/O facilities (\ie~\code{System.out.println} and \code{System.err.println}) \circled{1}.
%	The \CR tool displays the list of third-party libraries currently being included \circled{2}, and prompts a list of recommended libraries \circled{3}. The list of recommended libraries highlights the suggestion for using a logging library, \ie~\code{slf4j}\footnote{\url{https://www.slf4j.org/}} or \code{{commons-logging}\footnote{\url{https://commons.apache.org/proper/commons-logging/}}}. Then, a possible m igration to \code{slf4j} is shown \circled{4}, where the \code{System.out.println} and \code{System.err.println} invocations are replaced by \code{slf4j} \code{logger.info} and \code{logger.debug} calls, respectively.
%
%\begin{figure}[t!]
%	\centering
%	\includegraphics[wid\\th=\textwidth]{figs/CrossRecIDE.png}
%	%\vspace{-.3cm}
%	\caption{\CR IDE.}
%	\label{fig:IDE}
%	\vspace{-.3cm}
%\end{figure}
%
%
%
%In its current version, \CR recommends a set of libraries to developers, however it is likely that only some of the suggestions will result as useful for them. While the current version of \CR is unable to capture developers' feedback, it is possible to re-train again and obtain new recommendations based on the up-to-date code base.



\subsection{Combined use of \TFb and \MNB} \label{sec:combined_approach}

%So far, we have described \TF as a stand-alone recommender system by describing its constituent components. To highlight its flexibility in a different context, we propose an approach to improve \MNB. 

As already mentioned in Section~\ref{sec:Background}, though \MNB works in 
practice, it has its limitations. First, it can recommend only featured topics 
which account for a small fraction of all possible terms. Second, given that a 
repository already includes all suggested topics, \MNB is not able to recommend 
new ones. Moreover, the tool requires a \emph{balanced} dataset, \ie each topic 
must have a similar number of \RM files, and this is hard to come by in 
practice as topics are generally heterogeneous. With \TFb, we attempt to 
improve \MNB by combining it with the collabora\-tive-filtering technique 
presented in the previous subsection. The set of featured topics predicted by 
the \MNB model is used as input to feed \TFa, which then runs the filtering 
process to deduce the inclusion of new topics.

\begin{figure}[h!]
	\centering
	\includegraphics[width=\linewidth]{figs/entangled.pdf}
	\caption{Overview of the combined approach.}
	\label{fig:entangled}
\end{figure}

Figure~\ref{fig:entangled} depicts an overview of the combined use of \MNB and 
\TFb: a 
list of featured topics computed by \MNB using README files is fed as input for 
\TFa, which computes a list or ranked topics, including also non-featured ones. 
Finally, a list is generated by combining the topics computed by \TF with the 
top-N featured ones computed by \MNB, and recommended to developers.




%as follows. %\emph{entangle} our tool with the MNB network using it as a black box. 
%In particular, we show how \TF can be combined with other recommender systems that use a different mining strategy.

%As mentioned before, this recent work using the \RM file of a repository to predict featured topics. It involves all the standard techniques employed in the ML domain \ie textual engineering, feature extraction, and training phase. Given a \RM file, the approach computes vectors using the TF-IDF weighting scheme to extract features. Then, the model is trained to retrieve the most probable featured topics according to the multinomial distribution with the Naive Bayesian assumption. The outcomes are evaluated using the ten folder validation process.
%, resulting in \TFb

%The aim of this kind of analysis is to evaluate \TF capability using a well-founded technique in the literature.  can help improve \MNB 
We assume that \TFb is beneficial in the following aspects:
\begin{itemize}
	\item It is able to recommend non-featured topics, which are selected from similar repositories, independently from their nature;
	\item \TFb recommends topics by iterating over refining steps: once we select some topics from the recommended ones, \TFb can discover new topics using the selected ones as new input;
	\item \MNB does not provide additional recommendations given that the suggested topics are already included. As we will see in Section~\ref{sec:EXP1}, \TFb considerably improves the performance when more topics are incorporated as input.
\end{itemize}

%The experiments conducted 
In the following sections, we introduce the experiments conducted to evaluate 
the performance of \TFa, \MNB, and their combined use by comparing the 
accuracy and coverage metrics.



\section{Evaluation Materials and Methods}		
\label{sec:Evaluation}
%In the following, we 

%\color{blue}




%\revised{:q

In this section, we report on the materials and methods used to evaluate \TF. %evaluation of \CT, having the {\em goal} of evaluating the performance of the proposed approach. 
Section~\ref{sec:Dataset} presents the datasets exploited in the evaluation. Section~\ref{sec:methodology-metric} and Section~\ref{sec:metrics} describe the methodology and metrics, respectively. Finally, the research questions are highlighted in Section~\ref{sec:ResearchQuestions}.

To facilitate future research, we made available the \TF tool together with the related data in a GitHub repository\footnote{https://github.com/ESEM2020-TopFilter/TopFilter}.

%ten equal parts, so-called folds
%We also exploited 
%was also exploited 

\begin{figure*}[h!]
	\centering
	\includegraphics[width=0.9\linewidth,keepaspectratio]{figs/evaluationCF.pdf}
	%\vspace{-.3cm}
	\caption{Evaluation Process.}
	\label{fig:EvaluationProcess}
	\vspace{-.3cm}
\end{figure*}

\subsection{Data Extraction} \label{sec:Dataset}



%\subsection{Data Cleaner}  \label{sec:filter}

As a preprocessing, we filter the initial set of topics using their frequencies counted on the entire \GH dataset. We remove irrelevant topics to reduce the noise in the prediction phase. Through the \emph{cut-off} value, we progressively increase the frequency threshold to evaluate possible impacts on overall performances. As stated in \cite{repo-topix}, this preprocessing can improve the final results, thus we decide to apply it as a first step. 

To this end, we develop tailored Python scripts that apply this filter to the initial dataset. As a \GH user can manually specify the topic list for his repository, a lot of them can contain infrequent or improper terms \ie the name of the author, duplicated values, terms that rarely appear to name a few. On one hand, imposing such preprocessing reduces the repositories to analyze as well as topics to recommend. On the other hand, we improve the overall quality of recommendation by pruning "bad" terms. 
This pruning phase is computed offline and doesn't affect the time required for the recommendation process. 

%Before encoding topics in the repo-topic ratings matrix, raw topics are pre-processed removing possible syntactical duplicates terms (\eg \textit{document} and \textit{documents}). More specifically, stemming, lemmatization, and stop words removal Natural Language Processing (NLP) techniques have been applied on the mined topics.




%\color{blue}

To evaluate the approach, we reuse the same dataset employed for the \MNB available here \cite{MNBreplication}. The \GH query language \cite{understanding} allows the fetching of relevant repository metadata including name, owner, and list of topics to mention a few. Thus, we \emph{randomly} collected a dataset consisting of 6,258 repositories that use 15,757 topics by means of the GitHub API \cite{pygithub/pygithub_2019}. We employ the \GH star voting mechanism as a popularity measure to avoid including unpopular, unmaintained and toy projects \cite{borges_whats_2018}. As claimed in several works\cite{borges_popularity_2017, borges_predicting_2016}, a high number of stars means the attention of the community for that project. So, we impose the following filter during the query execution:
\begin{equation}
\small
Qf = "is:featured \; topic:t \; stars:100..80000 \; topics:>=2"
\end{equation}%
to consider only \GH repositories having a number of stars between 100 and 80,000, and tagged with at least two topics. The boolean qualifier \emph{is:featured} is used in the \MNB work to group repositories given a certain featured topic (please refers to \url{https://github.com/topics} for the complete list of featured topics). As \CT is able to retrieve both featured and not-featured topics, this filter doesn't affect the quality of the collected data.
To investigate the \CT prediction performances, we populated five different datasets starting from the original one by varying the topic frequency cut-off value \emph{t} \ie the maximum frequency of the topic distribution (it will be better described in Section~\ref{sec:methodology-metric}). In this way, we remove the infrequent elements from the dataset to analyze the impacts on the recommendation phase as well as on the composition of the dataset. Table \ref{tab:Datasets} summarizes the datasets' features with \emph{t} = 1, 5, 10, 15, 20. In particular, for each dataset the \emph{Avg. topics} values are the average number of topics that the repositories include, and the  \emph{Avg. freq. topics} values are the average frequency of the topics in the dataset, \ie the average occurrences of the topics among  the considered repositories in the dataset.


%\begin{table}[h]
%\centering
%\resizebox{8.5cm}{!} {
%\begin{tabular}{|l|l|l|c|c|}
%\hline
%\textbf{Dataset} & \textbf{No. of repos} &\textbf{ No. of topics} & \textbf{Avg topics for repo} & \textbf{Avg freq. for topic} \\ \hline
%D$_1$  &       6,253      &    15,743       &  9.9  &  3.9       \\ \hline
% D$_5$  &        3,884      &    1,989      &     8.4  &   16.5   \\ \hline
%D$_{10}$  &    2,897         &      964	     &   8.0    &  24.1 \\ \hline
%D$_{15}$  &    2,273        &   634       &   7.8  &  28.1       \\ \hline
%D$_{20}$  &    1,806       &   456        &   7.7 &  30.5       \\ \hline
%
%\end{tabular}
%}
%\caption{Datasets' description.}
%\label{tab:datasets}
%\end{table} 


\begin{table}[h!]
%	\small
	\caption{Datasets.}
	\begin{tabular}{|l|p{0.68cm}|p{0.68cm}|p{0.68cm}|p{0.68cm}|p{0.68cm}|} \hline
		 & \textbf{ D$_{1}$} & \textbf{D$_{5}$} & \textbf{ D$_{10}$} & \textbf{D$_{15}$} & \textbf{D$_{20}$} \\ \hline
		Number of repos & 6,253 & 3,884 & 2,897  & 2,273 & 1,806  \\ \hline
		Number of topics & 15,743 & 1,989 & 964 & 634 & 456 \\ \hline
		Avg. topics & 9.9 & 8.4 & 8.0  & 7.8 & 7.7 \\ \hline
		Avg. freq. topics & 3.9  & 16.5  & 24.1  & 28.1  & 30.5  \\ \hline
	\end{tabular}	
	\label{tab:Datasets}	
\end{table}



%<<<<<<< HEAD
As we can see in the next section, removing the infrequent topics improves the overall quality of the considered datasets. 
Similarly to the other collaborative filtering approaches, the overall prediction performance strongly depends on the dataset. As we will demonstrate in the next section, the collaborative filtering provides better prediction performance when there are enough data (\ie topics) in the training set to resemble the repository behaviour. After infrequent topics is removed, the repository that consist of less then 5 topics are filter out from the dataset because they contain very few information to enable the collaborative filtering prediction. In particular, we remove around 2,300 repositories by increasing the cut-off value from 1 to 5. It means that the excluded repositories in Dataset D$_5$ are tagged with topics that rarely appear in the considered repositories. This finding is strengthened by the number of topics, which dramatically decreases to 1,989. The other datasets confirm this trend even though the delta of removed repositories goes down at each filtering step. Thus, we stop at t=20 and consider Dataset D$_{20}$ as the best one according to our metrics. Additionally, we observe that repositories are tagged by 9.9 and 7.7 topics on average for \emph{t} = 1 and \emph{t} = 20 respectively. This demonstrates that a huge number of topics doesn't help the discoverability of a project.

\begin{figure*}[h!]
\centering
	\begin{tabular}{c c }	
	\subfigure[Dataset D$_5$]{\label{fig:dt5_stars}
	\includegraphics[width=0.45\linewidth]{figs/dataset_5_stars.pdf}} &
	\subfigure[Dataset D$_{10}$]{\label{fig:dt10_stars}
	\includegraphics[width=0.45\linewidth]{figs/dataset_10_stars.pdf}} \\
\subfigure[Dataset D$_{15}$]{\label{fig:dt15_stars}
	\includegraphics[width=0.45\linewidth]{figs/dataset_15_stars.pdf}} & 	
\subfigure[Dataset D$_{20}$]{\label{fig:dt20_stars}
	\includegraphics[width=0.45\linewidth]{figs/dataset_20_stars.pdf}}\\	
	\end{tabular} 
	\caption{Quality analysis of the examined datasets.}
	\label{fig:comparison}
\end{figure*}

 
%=======
%As we can see, removing the infrequent topics improves the overall quality of the considered datasets. In particular, we remove around 2,300 repositories by increasing the cut-off value from 1 to 5. It means that the excluded repositories in Dataset D$_5$ are tagged with topics that rarely appear in the considered repositories. This finding is strengthened by the number of topics, which dramatically decreases to 1,989. The other datasets confirm this trend even though the delta of removed repositories goes down at each filtering step. Thus, we stop at t=20 and consider Dataset D$_{20}$ as the best one according to our metrics. Additionally, we observe that repositories are tagged by 7 topics on average. This demonstrates that a huge number of topics doesn't help the discoverability of a project. 
%>>>>>>> 2241e0e812c956f441252b061dd0d75d6f6927ea
Furthermore, we evaluate the quality of the OSS project belonging to the examined dataset. As mentioned before, the \GH community assesses this aspect by mainly using forks and stars. Thus, we collect this data for each dataset using the same Github API library employed for the crawling. Figure \ref{fig:comparison} shows the comparison among all the examined datasets. As can see, filtering repositories by the \emph{t} value helps to smooth the distribution. On one hand, the Dataset D$_5$ contains more repositories with a high forks number rather than the ultimate dataset \ie it reaches around 20,000 forks against 15,000 with t=5 and t=20 respectively. On the other hand, the slope depicted in Dataset D$_{20}$ is higher than the original dataset. The positive trend is confirmed by observing the distribution of the other datasets \ie Dataset D$_{10}$ and D$_{15}$. In particular, we are able to remove repositories with less number of stars \ie from 5,000 to 10,000. 
From this study, we observe a correlation between the number of stars and the topics' frequency. In other words, most frequent topics appear in the high-ranked repositories and this finding affects the quality of the recommendations. 







%Among these projects, only $7$ of them have been forked from other projects. Such original projects have been excluded from the dataset as their forked ones share highly similar libraries, and this may introduce bias in the recommendation outcomes. We represent the distribution of projects with respect to the number of forks, commits and pull requests in Fig.~\ref{fig:ForkCommitPull}. Most projects have a low number of pull requests, \ie lower than 100, however many of them have a large number of forks and commits. Forking is a means to contribute to the original repositories~\cite{Jiang:2017:WDF:3042021.3042043}. Furthermore, there is a strong correlation between forks and stars~\cite{7816479}, as it is further witnessed in Fig.~\ref{fig:ForkStarIssue}. A project with a high number of forks means that it gets attention from the OSS community. In this sense, having many forks can be considered as a sign of a well-maintained and well-received project. Meanwhile, as commits have an impact on the source code~\cite{8009930}, the number of commits is also a good indicator of how a project has been developed.
%}


%e mined dependency specification by means of \code{code.xml} or \code{.grad\-le} files.\footnote{The files \code{pom.xml} and with the extension \code{.gradle} are related to management of dependencies by means of Maven (\url{https://maven.apache.org/}) and Gradle (\url{https://gradle.org/}), respectively.} Fig.~\ref{fig:NumOfLibsD1} depicts the distribution of libraries across the projects. Most of the libraries in \code{D1} ($12,962$) are used by a small number of projects, and only $10$ libraries are extremely popular by being included in more than $200$ projects. By carefully investigating the dataset, we also see that most projects contain a small number of dependencies, \ie $48\%$ of the projects include less than $20$ libraries and just $15\%$ of them include more than $100$ libraries. \}


%\color{blue}
\subsection{Evaluation process}\label{sec:methodology-metric}
The \emph{ten-fold cross-validation} methodology~\cite{kohavi1995study} has been used to assess the performance of \CT, \MNB and combined approach where every time $9$ folds 
%(each one contains $625$ projects) 
are used for training and the remaining one for testing.
For each testing project \emph{p}, we randomly delete half topics and save it as ground truth (\emph{GT(p)}) . The ground truth data  will be used to validate the recommendation outcomes. The remaining half topics are used as query topics to the \CT.
Figure~\ref{fig:EvaluationProcess} depicts the evaluation process consists of three consecutive phases,~\ie~\emph{Data Preparation},~\emph{Recommendation}, and~\emph{Outcome Evaluation}.

\noindent
$\blacksquare$~\textbf{Data Preparation phase} collects repositories that match the requirements defined in previous section from GitHub during \code{Data collection} step. This dataset is used to evaluate \CT, \MNB, and the combination of two. 
The dataset is then split into training and testing sets (\ie \code{ Split ten-fold} activity).
Due to the different nature of the the recommender systems (\ie \MNB requires \code{README} files as input and training data, whereas \CT uses a set of assigned topics as input and for training the recommendation system), the testing and training data are specifically cooked for both approaches).
The \code{Split topic} activity resembles a real development process where a developer has already included some topics in his repository (\ie \code{Query topics}) and waits for recommendations \ie additional topics to be incorporated. \CT recommender system is expected to provide her with the other half, \ie \code{ GT topics}.

\noindent
$\blacksquare$~\textbf{Recommendation phase} follows three different flows, according to the required input and produced output of the three mentioned approaches. In particular, the common operations are in white while the three different evaluation flows are represented in a grayscale fashion (\ie light grey, grey and dark grey boxes are related to \code{\CT}, \code{\MNB}, and \code{Entangled} approaches evaluation respectively).
To enable \CT, we extract a portion of topics from a given testing project \ie the ground-truth part (it is defined as $GT(p)$ in the following). The left part is used as a query to produce recommendations (see the dotted line flow). As the \MNB uses the README file of a repository to predict a set of topics, this doesn't require any topic as input. Thus, the approach encodes the document relevant information in vectors using the TF-IDF weighting scheme. Then, to feed the network that delivers a set of topics (see the bold line). Finally, the entangled approach uses \CT as the recommendation engine which is fed by the \MNB suggested topics (see dashed line flow). In this respect, both \code{Testing data} and \code{Training set} boxes are simplified to provides the needed data (\ie README file and assigned topics) to the different recommender systems.

\noindent
$\blacksquare$~\textbf{Outcome Evaluation phase} evaluate the recommendation results with those stored as ground-truth data to compute the quality metrics (\ie \code{Success rate}, \code{Precision}, \code{Recall}, and \code{Catalog coverage}) during the \code{Comparison} activity.  

It is worth noting that we can't directly compare directly \CT and \MNB approaches because they rely on different input data (\ie \CT requires an initial set of assigned topics for suggesting new ones, whereas\MNB uses the information mined from the README files to recommend the expected topics). 
%, which are called \emph{\textbf{te}}, and serve as input for \code{Similarity Computation} and \code{Recommendation} components.


\subsection{Metrics' definition}\label{sec:metrics}

In the context of recommendation systems, several metrics have been proposed to evaluate a ranked list of recommended items \cite{DBLP:conf/rweb/NoiaO15}. In the scope of this paper, \emph{success rate} \emph{accuracy}, and \emph{catalog coverage} have been used to study the systems' performance as already proposed in \etal~\cite{Robillard:2014:RSS:2631387}

% and other studies~\cite{6671293},\cite{Nguyen:2015:ESP:2740908.2742141}.
%Before defining the metrics involved in the experiment, we present the following notations to make easier the understanding of the considered metrics:
First of all, we present the following notations to make easier the understanding of the metrics involved in our experiment:
\begin{itemize}[noitemsep,topsep=0pt]
	\item \emph{t} is the frequency cut-off value of input topics (\ie all topics that occur less than \emph{t} times are removed from the dataset)
	\item |\emph{t$_{in}$}| is the size of topics that \CT takes as input;
	\item \emph{N} is the cut-off value for the recommended ranked list of topic;% of recommended libraries;
	\item \emph{k} is the number of top-similar neighbour projects \TF considers to predict suggested topics;
	\item \emph{GT(r)} is defined as the half part of the extracted topics for a testing project \emph{r}, and \emph{GT(r)} have been used as ground-truth data in the evaluation process;
%	\item For a testing project \emph{r}, a half of its topics are extracted and used as the ground-truth data named as \emph{GT(r)};
	\item $REC_{N}(r)$ is the outcome of \TF to a given repository \emph{r}, \ie the top-\emph{N}  suggested topics in a descending order by considering the recommendation scores;
	%\item $REC(r)$ is the \emph{top-N} topics recommended to a repository \emph{r}. It is a ranked list in descending order of real scores;
	\item a recommended topic \emph{rt} to a repository \emph{r} is marked as a \emph{match} if $rt \in REC(r)$;
	\item  $match_{N}(r)$ is the set of items in $REC_{N}(r)$ that match with those in \emph{GT(r)} for a given a repository \emph{r}.
\end{itemize}



\noindent By using the such notations, the accuracy, success rate and coverage  metrics are defined as follows.  

\JDR{Rephrase this section}

\vspace{.1cm}
\noindent\textbf{Success rate@N.} Success rate is defined as the ratio of queries that have at least a matched topic among the total number of attempts. In particular, given a set of testing projects \emph{P}, this metric measures the percentage of \emph{top-N} query results that match at least one topic in the \emph{GT(p)} among the number of query project $p \in P$~\cite{6671293}: %It is formally defined as given below:
\vspace{-.1cm}

\begin{equation} \label{eqn:RecallRate}
success\ rate@N=\frac{ count_{p \in P}( \left | match_{N}(p) \right | > 0 ) }{\left | P \right |} %\times 100\%
%success\ rate@N=\frac{ count_{p \in P}( \left | GT(p) \bigcap (\cup_{r=1}^{N} REC_{r}(p)) \right | > 0 ) }{\left | P \right |}
\end{equation}

\vspace{.1cm}
\vspace{.1cm}




%\noindent \JDR{Check if it will be used} \textbf{Success rate$_M$@N.} Given a set of testing projects \emph{P}, this metric measures the rate at which a recommender system returns at least $M$ topics match among \emph{top-N} items for every project $p \in P$ \cite{6671293}: %It is formally defined as given below:
%\vspace{-.1cm}
%
%\begin{equation} \label{eqn:RecallRate}
%success\ rate_M@N=\frac{ count_{p \in P}( \left | match_{N}(p) \right | >= M ) }{\left | P \right |} %\times 100\%
%%success\ rate@N=\frac{ count_{p \in P}( \left | GT(p) \bigcap (\cup_{r=1}^{N} REC_{r}(p)) \right | > 0 ) }{\left | P \right |}
%\end{equation}
%
%\vspace{.1cm}
%\noindent where the function \emph{count()} counts the number of times that the boolean expression specified in its parameter is \emph{true}.

\noindent \textbf{Accuracy.} Because of \emph{success rate@N} only measures successful queries and it does not reflect how accurate the outcome of a recommender system is
(\eg considering two recommendation results where the former counts $1$ topic match out of $5$ and the latter that returns all $5$ topic matches,  both results equally affect the success rate value because both results include at least one matched topic.), accuracy has been considered as further quality indicator for \TF. Given a list of \emph{top-N} libraries, \emph{precision@N} and \emph{recall@N} are utilized to measure the \emph{accuracy} of the recommendation results. \emph{precision@N} is the ratio of the \emph{top-N} recommended topics belonging to the ground-truth dataset, whereas \emph{recall@N} is the ratio of the ground-truth topics appearing in the \emph{N} recommended items \cite{Nguyen:2019:FRS:3339505.3339636},\cite{DiNoia:2012:LOD:2362499.2362501},\cite{Davis:2006:RPR:1143844.1143874}:

%However, \emph{success rate@N} does not reflect how accurate the outcome of a recommender system is. For instance, given only one testing project, there is no difference between a system that returns $1$ topic match out of $5$ and another system that returns all $5$ topic matches, since \emph{success rate@5} is $100\%$ for both cases (see Eq.~\eqref{eqn:RecallRate}). Thus, given a list of \emph{top-N} libraries, \emph{precision@N} and \emph{recall@N} are utilized to measure the \emph{accuracy} of the recommendation results. \emph{precision@N} is the ratio of the \emph{top-N} recommended topics belonging to the ground-truth dataset, whereas \emph{recall@N} is the ratio of the ground-truth topics appearing in the \emph{N} recommended items \cite{Nguyen:2019:FRS:3339505.3339636},\cite{DiNoia:2012:LOD:2362499.2362501},\cite{Davis:2006:RPR:1143844.1143874}: %,Nguyen:2015:CRV:2942298.2942305 %}. A concrete definition for the metrics is given below \cite{

%\vspace{-.2cm}
% \cite{Saracevic:1995:EEI:215206.215351}

\begin{equation} \label{eqn:Precision}
precision@N = \frac{ \left | match_{N}(p) \right | }{N}
%precision@N(p) = \frac{\sum_{r=1}^{N}\left | GT(p) \bigcap REC_{r}(p) \right |}{N}
\end{equation}
%\vspace{-.1cm}
\begin{equation} \label{eqn:Recall}
recall@N = \frac{ \left | match_{N}(p) \right | }{\left | GT(p) \right |}
%recall@N(p) = \frac{\sum_{r=1}^{N}\left | GT(p) \bigcap REC_{r}(p) \right |}{\left | GT(p) \right |}
\end{equation}
%\vspace{-.1cm}

\vspace{.1cm}
\vspace{.1cm}



\noindent \textbf{Catalog coverage.} This metric is particularly suitable to measure the performance predictions of recommendation systems that suggest a list of items \cite{ge_beyond_2010}. Given the set of projects $U_p$, we compare the number of recommended topics with the global number of the available ones \ie $ REC_{N}(p)$ and T respectively. Reversely from the previous two metrics, the Catalog Coverage measures the suitability of the delivered topics considering all the possible set of values. From the evaluation point of view, it is interesting to assess the impact of N value on the coverage stability, meaning what values of N impacts on the overall prediction performances. 



\begin{equation}\label{eqn:Coverage}
%coverage = \frac{\left | \cup_{p\in P} \left [  \cup_{r=1}^{N} REC_{r}(p) \right ] \right | }{\left | L \right |} 
coverage@N = \frac{\left | \cup_{p\in P} REC_{N}(p) \right | }{\left | T \right |} 
\end{equation}

\vspace{.1cm}
\vspace{.1cm}





%To assess the performance of \CR proposed approach, we applied ten-fold cross-validation, considering every time $9$ folds (each one contains $625$ projects) for training and the remaining one for testing. 
%	For every testing project \emph{p}, a half of its topics are \emph{randomly} 
%	taken out and saved as ground truth data, let us call them \emph{GT(p)}, which will be used to validate the recommendation outcomes. The other half are used 
%	as testing libraries or query, which are called \emph{\textbf{te}}, and serve 
%	as input for \code{Similarity Computation} and \code{Recommendation}.
%The splitting mimics a real development process where a 
%developer has already included some topics in the current project, \ie 
%\emph{\textbf{te}} and waits for recommendations, that means additional topics to be incorporated. A recommender system is expected to provide her 
%with the other half, \ie \emph{GT(p)}. %To ensure a reliable comparison between \LR and \CR, we performed cross-validation for both using exactly the same folds.
%
%
%
%There are several metrics available to evaluate a ranked list of recommended items \cite{DBLP:conf/rweb/NoiaO15}. In the scope of this paper, \emph{success rate}, \emph{accuracy}, \emph{sales diversity}, and \emph{novelty} have been used to study the systems' performance as already proposed by Robillard \etal~\cite{Robillard:2014:RSS:2631387} and other studies~\cite{6671293},\cite{Nguyen:2015:ESP:2740908.2742141}. For a clear presentation of the metrics considered during the outcome evaluation, let us introduce the following notations:
%
%\begin{itemize}[noitemsep,topsep=0pt]
%	\item \emph{N} is the cut-off value for the ranked list;% of recommended libraries;
%	\item \emph{k} is the number of neighbor projects exploited for the recommendation process;
%	\item For a testing project \emph{p}, a half of its libraries are extracted and used as the ground-truth data named as \emph{GT(p)};
%	\item $REC(p)$ is the \emph{top-N} libraries recommended to \emph{p}. It is a ranked list in descending order of real scores;%, with $REC_r(p)$ being the recommended library in the position $r$.		
%	\item If a recommended library $l \in REC(p)$ for a testing project $p$ is found in the ground truth of $p$ (\ie \emph{GT(p)}), hereafter we call this as a library \textit{match} or \textit{hit}.	
%\end{itemize}	
%
%
%
%If $REC_{N}(p)$ is the set of top-$N$ items and $match_{N}(p)=  GT(p) \bigcap REC_{N}(p) $ is the set of items in the \emph{top-N} list that match with those in the ground-truth data, then the metrics are defined as follows.
%
%\paragraph{\textbf{Success rate@N}} Given a set of testing projects \emph{P}, this metric measures the rate at which a recommender system returns at least a topic match among \emph{top-N} items for every project $p \in P$ \cite{6671293}: %It is formally defined as given below:
%\vspace{-.3cm}
%
%\begin{equation} \label{eqn:RecallRate}
%success\ rate@N=\frac{ count_{p \in P}( \left |  match_{N}(p) \right | > 0 ) }{\left | P \right |} %\times 100\%
%%success\ rate@N=\frac{ count_{p \in P}( \left |  GT(p) \bigcap (\cup_{r=1}^{N} REC_{r}(p)) \right | > 0 ) }{\left | P \right |} 
%\end{equation}
%
%\noindent where the function \emph{count()} counts the number of times that the boolean expression specified in its parameter is \emph{true}.
%
%
%\paragraph{\textbf{Accuracy}} Accuracy is considered as one of the most preferred \emph{quality indicators} for Information Retrieval applications \cite{Saracevic:1995:EEI:215206.215351}. However, \emph{success rate@N} does not reflect how accurate the outcome of a recommender system is. For instance, given only one testing project, there is no difference between a system that returns $1$ topic match out of $5$ and another system that returns all $5$ topic matches, since \emph{success rate@5} is $100\%$ for both cases (see Eq.~\eqref{eqn:RecallRate}). Thus, given a list of \emph{top-N} libraries, \emph{precision@N} and \emph{recall@N} are utilized to measure the \emph{accuracy} of the recommendation results. \emph{precision@N} is the ratio of the \emph{top-N} recommended topics belonging to the ground-truth dataset, whereas \emph{recall@N} is the ratio of the ground-truth topics appearing in the \emph{N} recommended items \cite{Nguyen:2019:FRS:3339505.3339636},\cite{DiNoia:2012:LOD:2362499.2362501},\cite{Davis:2006:RPR:1143844.1143874}: %,Nguyen:2015:CRV:2942298.2942305 %}. A concrete definition for the metrics is given below \cite{
%
%%\vspace{-.2cm}
%% \cite{Saracevic:1995:EEI:215206.215351}
%
%\begin{equation} \label{eqn:Precision}
%precision@N = \frac{ \left |  match_{N}(p) \right | }{N}
%%precision@N(p) = \frac{\sum_{r=1}^{N}\left |  GT(p) \bigcap REC_{r}(p) \right |}{N}
%\end{equation}
%%\vspace{-.1cm}
%\begin{equation} \label{eqn:Recall}
%recall@N = \frac{ \left |  match_{N}(p) \right | }{\left | GT(p) \right |}	
%%recall@N(p) = \frac{\sum_{r=1}^{N}\left |  GT(p) \bigcap REC_{r}(p) \right |}{\left | GT(p) \right |}	
%\end{equation}
%%\vspace{-.1cm}
%



\subsection{Research Questions} \label{sec:ResearchQuestions}
By performing the evaluation, we aim at addressing the following research questions:
\begin{itemize}
	\item[--] \rqfirst To answer this question, we investigate different configurations to find the best one \ie we variate the number of input topics \emph{T}, the number of neighbours \emph{N} and the considered number of outcomes \emph{N}.
	
	%\item[--] \rqsecond Because of \CT and \MNB are completely different in term of input data, we are interested in comparing them by considering many factors that can impact on the performance.
	\item[--] \rqsecond From an empirical point of view, it is relevant to analyze the combination of the two approaches and measure its performances.
\end{itemize}


We study the experimental results in the next section by referring to these research questions.


\section{Experimental Results}
\label{sec:ExperimentalResults}
This section discusses the findings of the qualitative assessment. To address the formulated research questions, we perform two different experiments. Section \ref{sec:EXP1} discusses the \CT results by variating different parameters. % We measure the predict performances of the \MNB in Section \ref{sec:EXP2}. 
The results obtained with the entangled approach (\ie the combination of \CT and \MNB approaches) are investigate in Section~\ref{sec:EXP3} . 


\subsection{\CT evaluation} \label{sec:EXP1}
 \rqfirst
 
To find the best configuration in terms of prediction performances, we experiment with different \CT configuration by variating the available parameters \ie number of neighbors \emph{k}, the recommended topic cut-off value \emph{N}, and the involved dataset.   %The former refers to the number of similar repositories used in the recommendation engine. 
%The latter value \emph{t} is used to select the input topics based on their frequencies: given an initial set of topics, we filter them with the cut-off value to reduce the noise in the original dataset. Then, the recommendation phase is enabled by varying the number of presented parameters. According to Section \ref{sec:methodology-metric}, N is the cut-off value and \emph{k} is the number of neighbours of the graph. We evaluate different configuration by setting N=1,5,10,15,20 and k=5,10,15,20,25. 
%Figure \ref{fig:configs} shows the results  in terms of precision and recall. 


As we are relying on a collaborative filtering technique, the number of output topics, the number of neighbours, and the data preprocessing play an important role in the assessment. Thus, we variate  the recommended list of topics \emph{N} for 5 and 10, and the number of neighbours \emph{k} \ie \emph{N} = \{5, 10, 15, 20, 25\}. 
%we use different topic frequency cut-off \emph{t} to remove very infrequent topics from the dataset. 
The bar charts in Fig.~\ref{fig:success-rateN5} and \ref{fig:success-rateN10} show the average success rates of all ten folds of \CT.%divided by the different topic frequency cut-off \emph{t}
Both figures depict the results of \CT applied on the different datasets defined in Section~\ref{sec:Dataset} \ie $Dt_{1}$, $Dt_{5}$, $Dt_{10}$, $Dt_{15}$, and $Dt_{20}$.
In particular, Fig.~\ref{fig:success-rateN5} and Fig.~\ref{fig:success-rateN10} shows the success rate considering the first 5 and 10 recommended topics respectively. The horizontal axes shows the success rate outcomes for different size of neighbours \emph{N}.
Overall, it is evident that infrequent topics negatively affect both success rate values. At the first glance we can see that the success rate of \CT with all topics is much lower than others \emph{t} cut-off.
\begin{figure*}[t]
\centering
	\begin{tabular}{c c}	
	\subfigure[Success rate@5]{\label{fig:success-rateN5}
	\includegraphics[width=0.45\linewidth]{figs/successRateN@5.pdf}} &
	\subfigure[Success rate@10]{\label{fig:success-rateN10}
	\includegraphics[width=0.45\linewidth]{figs/successRateN@10.pdf}}
	
	\end{tabular} 
	\caption{Success rate with 5 and 10 input topics.}
	\label{fig:success5_10}
\end{figure*}
\begin{figure}[t!]
	\centering
	\includegraphics[width=.95\linewidth]{figs/PrecisionRecallCurve.png}
	\caption{Evaluation of the different configuration.}
	\label{fig:configs}
\end{figure}
The success rate assessment exhibits an average improvement of 10\% in all of the possible configurations obtained by variating \emph{N} and \emph{k} values. In particular, the success rate archives better results by setting higher values of \emph{k}. Nevertheless, increasing the number of neighbours gives remarkable benefits only until a certain threshold. Given \emph{k} = 5, the success rate@5 passes from 63\% to 69\% if we consider k=10. This positive delta decreases by augmenting the number of neighbours until it reaches a stable success rate. Thus, we can consider \emph{k} = 25 as the maximum value capable of improving prediction performances. This trend is further confirmed by introducing more topics in the initial set. We also demonstrate that the topic filtering preprocessing fosters this enhancement and noise removal is a critical step of the entire process.

This is also confirmed by the precision and recall curves depicted in Fig.~\ref{fig:configs}. 
%From the accuracy scores computed using Eq. (5) and Eq. (6), the Precision-Recall curves (PRCs) for all 10 rounds of validation and different values of k were sketched. 
The line graph depicts the precision and recall curves on average for all 10 rounds by considering \emph{N} value ranges from 1 to 20 and \emph{t}. So, each dot in a curve corresponds to a specific value of \emph{N}. 
These outcomes have been obtained by keeping 25 as the number of neighbours \emph{k} because we have already discussed that higher values of neighbours reach better prediction performances. Overall, the precision and recall values rise when the \emph{t} cut-off grows. Given that better prediction performance appears near to the upper right corner~\cite{DiNoia:2012:LOD:2362499.2362501}, the figure shows that a higher value of \emph{t} reaches better accuracy for all values of \emph{N}.

As defined in Section~\ref{sec:metrics}, the coverage metric is the percent of recommended topic in the training data that the model is able to recommend on a test set. Table~\ref{tab:coverage} reports the average coverage value for all ten rounds.
For each dataset (\ie $Dt_{1}$, $Dt_{5}$, $Dt_{10}$, $Dt_{15}$, $Dt_{20}$), 
the table consists of two coverage values: the former (\ie \emph{Dataset coverage}) is computed by considering the coverage with respect to the number of topics after the filtering step, whereas the latter considers all mined topics (\ie 15,743 topics). The value of \emph{Dataset coverage} grows by increasing the topic frequency cut-off value \emph{t}, whereas the \emph{Global coverage} decreases because there are no training data to recommend infrequent topics. In particular, the average \emph{Global coverage} values decrease from 9.306\% (Dt$_1$) to 1.805\% (Dt$_{20}$).
Having a higher value of \emph{t}, it strongly impacts on the global catalog coverage value, because too many training data are discarded due to the topic frequency cut-off \emph{t}. Differently from the discussed metric outcomes, this exeriment shows how an higher values of topic frequency cut-off negatively impacts on the global catalog coverage value.
The values of \emph{Dataset coverage} does not provide the information about the original topics directly mined from \GH.


\JDR{Change this part} In the methodology described in Section~\ref{sec:methodology-metric}, for each repository \emph{r}, the evaluation outcomes consider the half part of real topics as input and remaining ones as ground truth data \emph{GT(r)}. Because of we are also interested to understand how the number of input topics impacts on prediction performance, Fig.~\ref{fig:pr-input-topics} shows the average success rate of all ten folds by choosing different number of input topics. Varying |\emph{t$_{in}$}| means changing the length of input topics that enable the \CT collaborative filtering recommender. In this picture we report the average success of all folds values for the best configuration settings (\ie \emph{k} = 25 and \emph{t} = 20) . The success rate values exhibits an improvements when the size of input topic rises. This  behaviour demonstrate that \CT computes better similar repositories as neighbours when it has a higher number of topic as input. This is due to the similarity function that has been involved in the computation of first  \emph{k} neighbours. Because the average number of topics for each considered repository is 9.896 we can consider |\emph{t$_{in}$}| = 5 as the maximum value capable of improving prediction performances.
\begin{figure}[t!]
	\centering
	\includegraphics[width=\linewidth]{figs/sr_change_input_topics.pdf}
	\caption{Evaluation of the different input topics.}
	\label{fig:pr-input-topics}
\end{figure} 

%\begin{table*}[t]
%	\small
%\begin{tabular}{|l|p{1.2cm}p{1.2cm}|p{1.2cm}p{1.2cm}|p{1.2cm}p{1.2cm}|p{1.2cm}p{1.2cm}|p{1.2cm}p{1.2cm}|} \hline
%& \multicolumn{2}{c|}{Dt$_{1}$}                                                                                  & \multicolumn{2}{c|}{Dt$_{5}$}                                       & \multicolumn{2}{c|}{Dt$_{10}$}                                       & \multicolumn{2}{c|}{Dt$_{15}$}                                        & \multicolumn{2}{c|}{Dt$_{20}$}                                        \\ \hline
%
%\textbf{$N$}       & \textbf{Dataset coverage} & \textbf{Global coverage} & \textbf{Dataset coverage} & \textbf{Global coverage} & \textbf{Dataset coverage} & \textbf{Global coverage} & \textbf{Dataset coverage} & \textbf{Global coverage} & \textbf{Dataset coverage} & \textbf{Global coverage} \\ \hline
%2                & 2,313                       & 2,313                           & 11,339                      & 1,433                           & 17,558                       & 1,075                            & 21,991                       & 0,886                            & 24,220                       & 0,715                            \\ \hline
%4                & 3,925                       & 3,925                           & 18,698                      & 2,362                           & 28,634                       & 1,753                            & 35,766                       & 1,440                            & 38,681                       & 1,143                            \\ \hline
%6                & 5,494                       & 5,494                           & 25,583                      & 3,232                           & 38,317                       & 2,346                            & 46,130                       & 1,858                            & 50,044                       & 1,478                            \\ \hline
%8                & 7,075                       & 7,075                           & 31,940                      & 4,035                           & 46,296                       & 2,835                            & 54,265                       & 2,185                            & 58,791                       & 1,737                            \\ \hline
%10               & 8,720                       & 8,720                           & 37,899                      & 4,788                           & 52,904                       & 3,239                            & 61,027                       & 2,458                            & 65,011                       & 1,920                            \\ \hline
%12               & 10,385                      & 10,385                          & 43,314                      & 5,472                           & 59,044                       & 3,615                            & 67,093                       & 2,702                            & 70,484                       & 2,082                            \\ \hline
%14               & 12,073                      & 12,073                          & 48,436                      & 6,120                           & 64,249                       & 3,934                            & 72,385                       & 2,915                            & 75,275                       & 2,223                            \\ \hline
%16               & 13,872                      & 13,872                          & 53,261                      & 6,729                           & 68,852                       & 4,216                            & 76,682                       & 3,088                            & 79,187                       & 2,339                            \\ \hline
%18               & 15,753                      & 15,753                          & 57,402                      & 7,252                           & 73,080                       & 4,475                            & 80,553                       & 3,244                            & 82,659                       & 2,442                            \\ \hline
%20               & 17,746                      & 17,746                          & 61,497                      & 7,770                           & 76,738                       & 4,699                            & 83,649                       & 3,369                            & 85,363                       & 2,521                            \\ \hline
%\textbf{AVG} & \textbf{9,306}               & \textbf{9,306}                  & \textbf{37,495}              & \textbf{4,737}                  & \textbf{50,807}               & \textbf{3,111}                   & \textbf{58,089}               & \textbf{2,339}                   & \textbf{61,093}               & \textbf{1,805}                  \\ \hline
%\end{tabular}
%\caption{Coverage values for the proposed datasets (\ie D$_{1}$, D$_{5}$, D$_{10}$, D$_{15}$, D$_{20}$)}
%\label{tab:coverage}
%\end{table*}

\begin{table}[t!]
	\small
	\caption{Coverage values for D$_{1}$, D$_{5}$, D$_{10}$, D$_{15}$, D$_{20}$.}
	\begin{tabular}{|l|p{0.86cm}|p{0.86cm}|p{0.86cm}|p{0.86cm}|p{0.86cm}|}
		\hline
%		\multicolumn{1}{|c|}{\multirow{2}{*}{\textbf{N}}}                 & \multicolumn{5}{c|}{\textbf{Catalog Coverage}}                                                                                                                                                                                                                                                                                                                                          \\ \cline{2-6}
		
\textbf{N} & \textbf{ D$_{1}$} & \textbf{D$_{5}$} & \textbf{ D$_{10}$} & \textbf{D$_{15}$} & \textbf{D$_{20}$} \\ \hline
		2                & 2.313                                                                    & 1.433                                                                    & 1.075                                                                     & 0.886                                                                     & 0.715                                                                     \\ \hline
		4                & 3.925                                                                    & 2.362                                                                    & 1.753                                                                     & 1.440                                                                     & 1.143                                                                     \\ \hline
		6                & 5.494                                                                    & 3.232                                                                    & 2.346                                                                     & 1.858                                                                     & 1.478                                                                     \\ \hline
		8                & 7.075                                                                    & 4.035                                                                    & 2.835                                                                     & 2.185                                                                     & 1.737                                                                     \\ \hline
		10               & 8.720                                                                    & 4.788                                                                    & 3.239                                                                     & 2.458                                                                     & 1.920                                                                     \\ \hline
		12               & 10.385                                                                   & 5.472                                                                    & 3.615                                                                     & 2.702                                                                     & 2.082                                                                     \\ \hline
		14               & 12.073                                                                   & 6.120                                                                    & 3.934                                                                     & 2.915                                                                     & 2.223                                                                     \\ \hline
		16               & 13.872                                                                   & 6.729                                                                    & 4.216                                                                     & 3.088                                                                     & 2.339                                                                     \\ \hline
		18               & 15.753                                                                   & 7.252                                                                    & 4.475                                                                     & 3.244                                                                     & 2.442                                                                     \\ \hline
		20               & 17.746                                                                   & 7.770                                                                    & 4.699                                                                     & 3.369                                                                     & 2.521                                                                     \\ \hline
		\textbf{AVG} & \textbf{9.306}                                                           & \textbf{4.737}                                                           & \textbf{3.111}                                                            & \textbf{2.339}                                                            & \textbf{1.805}                                                            \\ \hline
	\end{tabular}
	\label{tab:coverage}
\end{table}


\begin{tcolorbox}[boxrule=0.86pt,left=0.3em, right=0.3em,top=0.1em, bottom=0.05em]
The quality evaluation demonstrates that \CT achieves better performance in terms of accuracy and success rate by increasing the number of considered neighbours \emph{k} and filtered data \emph{t}. %The number of neighbors and the topic filters contribute to this improvement. 
However, the conducted experiment shows that an higher values of topic frequency cut-off \emph{t} negatively impact on the global catalog coverage. 
For this reason, the \emph{t} values should be careful selected during the filtering phase to obtain balanced results in term of accuracy, success rate, and global catalog coverage.
However, the precision and recall values are still low, suggesting that bias lives in the users topic.
\end{tcolorbox}


%\subsection{\MNB evaluation} \label{sec:EXP2}

\rqsecond

Due to the lack of a baseline, we investigate the prediction performances of the \MNB to compare its outcomes with \CT. Reversely from the original paper, we apply the \MNB and we compared the outcomes whit respect to all topics (includin non featured ones) leaving the underlying structure untouched. This is necessary to undertake a fair comparison with \CT. Table \ref{tab:compareMNB} shows the evaluation results in terms  of the three aforementioned metrics. 


%\begin{table}[h]
%\centering
%
%
%\resizebox{8.5cm}{!} {
%\begin{tabular}{|l|l|l|l|l|l|l|}
%\hline
%  & \multicolumn{3}{c|}{ \textbf{\MNB}}          & \multicolumn{3}{c|}{ \textbf{\CT}}        \\ \hline
%\textbf{No. of input} & \textbf{Success rate} &\textbf{ Precision} & \textbf{Recall} & \textbf{Success rate} &\textbf{ Precision} &\textbf{ Recall} \\ \hline
%2  &       0.220       &    0.117       &  0.031       &     0.554         &      0.350     &   0.179      \\ \hline
% 4 &     0.392         &    0.119       &     0.063   &       0.682       &       0.267    &   0.271     \\ \hline
%6 &    0.538          &      0.122	     &   0.096     &     0.754         &    0.224       &   0.339     \\ \hline
%8 &    0.648          &  0.119         &   0.125      &         0.803     &     0.192      &   0.384     \\ \hline
%10 &      0.711        &    0.112       &   0.147     &      0.828        &   0.169        &    0.422    \\ \hline
%12 &     0.765         &      0.112     &   0.177     &        0.851      &     0.153      &     0.455   \\ \hline
%14 &      0.815        &    0.119       &   0.220     &    0.863          &   0.139         &   0.482     \\ \hline
%16 &       0.853       &     0.112      &     0.258   &        0.879      &   0.127        &   0.503     \\ \hline
%18 &      0.874        &     0.122      &     0.290   &       0.886       &     0.117      &    0.521    \\ \hline
%20 &     0.891         &    0.121       &    0.320    &      0.892        &   0.117        &       0.537 \\ \hline
%\rowcolor{Gray}
%\textbf{Average values} &    \textbf{0.651}        &   \textbf{ 0.120}       &   \textbf{ 0.165 }  &     \textbf{ 0.785}        &  \textbf{ 0.194}        &     \textbf{ 0.397 } \\ \hline
%\end{tabular}
%}
%\caption{Comparison of the two approaches.}
%\label{tab:compareMNB}
%\end{table} 
% Please add the following required packages to your document preamble:
% \usepackage[table,xcdraw]{xcolor}
% If you use beamer only pass "xcolor=table" option, i.e. \documentclass[xcolor=table]{beamer}
\begin{table}[]
	\scriptsize
	\resizebox{8.5cm}{!} {
	\begin{tabular}{|l|l|l|l|l|l|l|}
		\hline
		\rowcolor[HTML]{C0C0C0} 
		& \multicolumn{2}{c}{\textbf{Success rate}}         & \multicolumn{2}{|c|}{\textbf{Precision}}            & \multicolumn{2}{c|}{\textbf{Recall}}               \\ \hline
	 \rowcolor[HTML]{C0C0C0} 
		\textbf{\emph{N} } & \textbf{MNB} & \textbf{\CT }& \textbf{MNB} & \textbf{\CT} & \textbf{MNB} & \textbf{\CT }\\ \hline
		2                       & 0.220               & 0.554              & 0.117               & 0.350              & 0.031               & 0.179              \\
		4                       & 0.392               & 0.682              & 0.119               & 0.267              & 0.063               & 0.271              \\
		6                       & 0.538               & 0.754              & 0.122               & 0.224              & 0.096               & 0.339              \\
		8                       & 0.648               & 0.803              & 0.119               & 0.192              & 0.125               & 0.384              \\
		10                      & 0.711               & 0.828              & 0.112               & 0.169              & 0.147               & 0.422              \\
		12                      & 0.765               & 0.851              & 0.112               & 0.153              & 0.177               & 0.455              \\
		14                      & 0.815               & 0.863              & 0.119               & 0.139              & 0.220               & 0.482              \\
		16                      & 0.853               & 0.879              & 0.112               & 0.127              & 0.258               & 0.503              \\
		18                      & 0.874               & 0.886              & 0.122               & 0.117              & 0.290               & 0.521              \\
		20                      & 0.891               & 0.892              & 0.121               & 0.117              & 0.320               & 0.537              \\ \hline
		\rowcolor[HTML]{C0C0C0} 
		\textbf{Average} & \textbf{0.651}      & \textbf{0.785}     & \textbf{0.120}      & \textbf{0.194}     & \textbf{0.165}      & \textbf{0.397}    \\ \hline
	\end{tabular}}
\caption{Comparison of the two approaches.}
\label{tab:compareMNB}
\end{table}
We evaluate both approaches by variating the number of recommended topics up to 20. For the sake of the presentation, we report half of the data as we aim to show the overall trend.
As we can see, \CT outperforms the \MNB considering all the metrics. In particular, the success rate grows according to the number of input for both of the approaches. Although the \MNB reaches the same values of \CT with 20 input topics, the latter starts from an initial success rate value of 55\%. This statement holds for all metrics considered in the comparison. A significant achievement is given by the recall value which is the almost triplicated on average using \CT as the recommendation engine. For some input, the \MNB slightly outperforms \CT even though they are meaningless compared to the other findings. 
This gap is explained by the \MNB model features. In this comparison, we have added the not featured topics to the possible set of outputs \footnote{Due to the space issues, we cannot explain in detail the \MNB internal construction. Thus, the interested reader can find more information in the related work}. Consequently, the accuracy of the model is compromised by these new possible outcomes that the \MNB is not able to provide. This impacts especially on the recall values, as proved by the experiment. The aim of this comparison is to prove the soundness of \CT as a recommendation algorithm where the possible outcomes are heterogeneous \ie featured topics are shuffled with not featured ones. However, the accuracy is very low compared with the success rate. This could be affected by the similarity function embedded in the recommendation engine. 


\begin{tcolorbox}[boxrule=0.86pt,left=0.3em, right=0.3em,top=0.1em, bottom=0.05em]
From the evaluation, we can claim that \CT outperforms the \MNB. This result is lead by the construction differences between the two approaches, even though the \MNB performances are negatively affected by the introduction of the not featured topics. This demonstrates the rightness of \CT in a miscellaneous environment. 
\end{tcolorbox}

\subsection{Entangled evaluation} \label{sec:EXP3}
\rqsecond

Due to the internal construction of the \MNB, the direct comparison of the two approaches can bring biased results. Thus, we combined the two approaches to investigate potential improvements. We create this \emph{entagled} configuration by feeding \CT with the results of the \MNB. This simulates the exact use case of the collaborative filtering approach, in which the developer is represented by the \MNB. Table \ref{tab:combined} summarizes the results of this experiment by comparing the \MNB and the entangled approach. From the previous assessment, we figured out that \CT reaches best results considering the Dataset $Dt_{20}$. Thus, we choose this one to conduct this second evaluation. \CDS{Check if it is enough, maybe we can extend the evaluation to another additional dataset}
For experiment purposes, we variate the number of recommendation items as well as the number of input topics \ie \emph{Out} and \emph{Tin} values respectively. From the previous assessment, we figured out that the number of inputs leading the best results is Tin=5. Thus, we compare the outcomes considering the minimum number of input topics provided by the \MNB, \ie Tin=2. The results demonstrate that the \MNB gains notable improvement by means of the entangled configuration in terms of the mentioned metrics \ie accuracy, success rate, and catalog coverage. We witness that \CT outperforms the \MNB by augmenting the number of recommended items. 

In particular, after Out=8 the accuracy and success rate overcomes the \MNB results considering the \CT's best configuration even though the overall accuracy trend is decreasing. This happens because enlarging the set of recommended items impacts negatively on the precision values. Reversely, the success rate rises up to 0.855 with the best configuration of the entangled approach. As witnessed for the accuracy value, the \MNB records better results until a certain threshold of output items. This degradation in performance is due to the internal probabilistic model used by the approach. 

\begin{table*}[]
	\small
	\begin{tabular}{|l | lll| lll |lll |lll|}
		\hline
		& \multicolumn{3}{l|}{\textbf{Recall}} & \multicolumn{3}{l|}{\textbf{Precision}} & \multicolumn{3}{l|}{\textbf{Success rate}} & \multicolumn{3}{l|}{ \textbf{Catalog coverage}} \\ \hline
		$N$  & \textbf{MNB}     & \textbf{Tin=5}   & \textbf{Tin=2}  & \textbf{MNB}      & \textbf{Tin=5 }   & \textbf{Tin=2}   & \textbf{MNB}       & \textbf{Tin=5 }   & \textbf{Tin=2}    & \textbf{MNB}        & \textbf{Tin=5 }     & \textbf{Tin=2}      \\ \hline
		%1  & 0,018   & 0,018   & 0,018  & 0,138    & 0,138    & 0,138   & 0,136     & 0,136    & 0,136    & 10.769     & 4.835      & 4.835      \\ \hline
		2  & 0.035   & 0.031   & 0.031  & 0.206    & 0.118    & 0.118   & 0.363     & 0.217    & 0.217    & 9.068     & 8.593      & 8.593      \\ \hline
		%3  & 0,047   & 0,047   & 0,060  & 0,118    & 0,118    & 0,151   & 0,301     & 0,301    & 0,367    & 19.780     & 12.483     & 12.571     \\ \hline
		4  & 0.075   & 0.063   & 0.088  & 0.221   & 0.119    & 0.166   &  0.600     & 0.389    & 0.466    & 19.405     & 15.340     & 15.912     \\ \hline
		%5  & 0,081   & 0,081   & 0,104  & 0,124    & 0,124    & 0,157   & 0,476     & 0,476    & 0,508    & 25.494     & 18.307     & 19.054     \\ \hline
		6  & 0.094   & 0.121   & 0.119  & 0.187    & 0.153    & 0.149   & 0.635    & 0.601    & 0.549    &  24.682     & 22.131     & 21.780     \\ \hline
		%7  & 0,112   & 0,149   & 0,131  & 0,121    & 0,161    & 0,140   & 0,605     & 0,668    & 0,574    & 28.791     & 25.912     & 24.681     \\ \hline
		\rowcolor{Gray}
		8  & 0.106   & 0.171   & 0.142  & 0.159  & 0.162    & 0.133   &  0.680     & 0.704    & 0.599    & 27.967     & 29.296     & 27.428     \\ \hline
		%9  & 0,135   & 0,189   & 0,153  & 0,115    & 0,160    & 0,128   & 0,678     & 0,734    & 0,623    & 29.230     & 32.417     & 30.307     \\ \hline
		10 & 0.116   & 0.204   & 0.163  & 0.140    & 0.156    & 0.123   & 0.701     & 0.754    & 0.644    & 30.719     & 35.296     & 32.967     \\ \hline
		12 & 0.124   & 0.230   & 0.181  & 0.124    & 0.146    &  0.114   & 0.719     & 0.788    & 0.681    & 32.786     & 40.659     & 38.373     \\ \hline
		14 & 0.130   & 0.254   & 0.201  & 0.111    & 0.138    & 0,109   &  0.733     & 0.808    & 0.706    & 34.308     & 45.912     & 43.098     \\ \hline
		16 & 0.135   & 0.274   & 0.215  &  0.101    & 0.131    & 0.102   & 0.745    & 0.829    & 0.722    & 35.742     & 50.505     & 47.582     \\ \hline
		18 & 0.143   & 0.290   & 0.227  & 0.095   & 0.123    & 0.096   & 0.759     & 0.840    & 0.736    & 37.644     & 54.615     & 51.318     \\ \hline
		20 & 0.150   & 0.306   & 0.241  & 0.090    & 0.117    & 0.092   & 0.772     & 0.855    & 0.756    & 39.636    & 58.725     & 54.923    \\ \hline
	\end{tabular}
\vspace{.2cm}
\caption{Results for the entangled approach for the Dataset $Dt_{20}$.}
\label{tab:combined}
\end{table*}













%\begin{table}[h]
%\centering
%
%
%\resizebox{8.5cm}{!} {
%\begin{tabular}{|l|l|l|l|l|l|l|}
%\hline
%  & \multicolumn{3}{c|}{\textbf{\CT}}          & \multicolumn{3}{c|}{\textbf{Entangled approach}}        \\ \hline
%\textbf{No. of input} & \textbf{Success rate} &\textbf{ Precision} & \textbf{Recall} & \textbf{Success rate} &\textbf{ Precision} & \textbf{Recall} \\ \hline
%1  &       0.409       &    0.409       &  0.105       &     0.138         &      0.221     &   0.029      \\ \hline
% 2 &     0.554         &    0.350       &     0.179   &       0.220       &       0.198    &   0.053     \\ \hline
%3 &    0.632          &      0.301	     &   0.230     &     0.304         &    0.192       &   0.077     \\ \hline
%4 &    0.682          &  0.267         &   0.271      &         0.393    &     0.186      &   0.099     \\ \hline
%5 &      0.728        &    0.246       &   0.310     &      0.479        &   0.183        &    0.122    \\ \hline
%\rowcolor{Gray}
%6 &     0.754         &      0.224     &   0.339     &        0.983      &     0.278      &     0.225   \\ \hline
%7 &      0.778        &    0.207       &   0.363     &    0.999          &   0.340         &   0.322     \\ \hline
%8 &       0.803       &     0.192      &     0.384   &        1      &   0.371        &   0.40     \\ \hline
%10 &      0.828        &     0.169      &     0.422   &       1       &     0.382      &    0.511    \\ \hline
%15 &     0.872         &    0.132       &    0.493    &      1        &   0.322       &       0.636 \\ \hline
%20 &     0.892         &    0.117       &    0.537    &      1        &   0.266        &       0.696 \\ \hline
%\rowcolor{Gray}
%\textbf{Average values} &    \textbf{ 0.785}        &   \textbf{ 0.194}       &   \textbf{ 0.397}   &     \textbf{ 0.826 }       &  \textbf{ 0.296}        &       \textbf{0.433}  \\ \hline
%\end{tabular}
%}
%\caption{Results for the entangled approach.}
%\label{tab:combined}
%\end{table} 



Although the examined metrics are useful to analyze the overall performances, the catalog coverage can evaluate properly the capability to recommend a \emph{list} of items instead of a single one. Looking at the results, we can observe a substantial increase after 8 output items. As expected, the coverage dramatically increases with a larger number of outcomes for both of the considered approaches. Nevertheless, the positive gap of the entangled configuration is greater than the \MNB value. Considering the Out=20, the maximum value reached by the \MNB is 39.636 while the best configuration in the entangled experiment reaches a coverage of 58.725.

These findings can be explained by considering the nature of the considered topics. As said before, the \MNB can predict only featured topics as training the entire set of \GH topics is not possible due to the computation issues. Reversely, \CT covers a larger set of topics by enabling the described collaborative filtering technique. In this way, the \emph{entangled} is capable of suggesting both featured and not featured topics to the final user and enlarging the possible set of outcomes.



\begin{tcolorbox}[boxrule=0.86pt,left=0.3em, right=0.3em,top=0.1em, bottom=0.05em]
The entangled approach success in improving the prediction performances. By variating both the input and output number of topics, the accuracy and success rate experienced an enhancement even though the former reached low values. The \MNB lacks in catalog coverage, as clearly demonstrated by the higher value of the entangled experiment.
\end{tcolorbox}













\section{Threats to validity}
\label{sec:Threats}

This section discusses the threats that may affect the results of the evaluation. We also list the countermeasures taken to minimize these issues.

The \emph{internal validity} could be compromised by the dataset features, \ie the number of projects for each topic, the number of outcomes. We tackle this issue by variating the aforementioned parameters to build datasets with different characteristics. In this way, several scenarios are used to evaluate \TF's overall performances.

\emph{External validity} concerns the rationale behind the selection of the \GH repository used in the assessment. As stated in the related section, we download randomly repositories by imposing the quality filter on the stars. Nevertheless, some repositories could be tagged with topics that can affect the quality of the graph computed in the data extraction phase. To be concrete, a user can label its repository using terms that are not enough descriptive \ie using infrequent or duplicated terms in the topic list. To deal with this issue, we apply the topic filter as stated in Section \ref{sec:filter} to reduce the possible noise during the graph building.

Threats to \emph{construction validity} concerns the choice of \MNB as the baseline in the conducted experiment. First of all, the availability of the replication package allows a more comprehensive evaluation rather than other approaches. As we claimed before, the two approaches are strongly different from the construction point of view including the recommendation engine and data extraction components. To make the comparison as fair as possible, we run \MNB on the same datasets by adapting the overall structure for the ten folder validation.

\section{Related Work}
\label{sec:RelatedWorks}
This section discusses both \emph{(i)} approaches based on collaborative filtering techniques in recommending activity and \emph{(ii)} works that mine \GH projects. 

\subsection{Recommends item by means of collaborative filtering}
Amazon \cite{Linden:2003:ARI:642462.642471} proposes an item-to-item recommendation system to suggest relevant products to the final user.

\subsection{Recommending OSS using \GH topics}


\section{Conclusions and Future Work}
\label{sec:Conclusions}
\GH is nowadays the most popular platform to handle and maintain OSS projects. Topics have been introduced in 2017 to promote the project's visibility on the platform. Although a couple of works face the problem, there are additional challenges to be faced. 
In this work, we have presented \CT, a collaborative filtering based recommender system to suggest \GH topics.  By representing repositories and related topics in a graph format, we built a user-item matrix and apply a syntactic-based similarity function to predict missing topics. To assess the prediction performances, we compared \CT with a well-founded work based on an ML technique in terms of success rate and accuracy. The results show that \CT outperforms the opponent with a relevant improvement of the mentioned metrics. Furthermore, we combined the two approaches in an \emph{entangled} evaluation to explore possible enhancements. We figured out that \CT gained a significant boost in prediction performances by employing the \MNB outcomes as input topics. Nevertheless, the accuracy didn't reach higher values in all the experimental settings. To our best knowledge, it depends on the similarity function used in the recommendation engine as well as on the heterogeneity of the dataset. Thus, we are planning to extend \CT by adding different degrees of similarity \ie semantic analysis on topics, README encoding to name a few. Moreover, we can enlarge the evaluation by considering other common metrics in the collaborative filtering domain such as sales diversity and novelty. These augmentations will be considered as possible future work.



\section*{Acknowledgements}
The research described in this paper has been carried out as part of the CROSSMINER Project, which has received funding from the European Union's Horizon 2020 Research and Innovation Programme under Grant 732223.

%%
%% The next two lines define the bibliography style to be used, and
%% the bibliography file.
\bibliographystyle{ACM-Reference-Format}
\bibliography{ESEM2020}

%%
%% If your work has an appendix, this is the place to put it.
\end{document}
\endinput
%%
%% End of file `sample-sigconf.tex'.
