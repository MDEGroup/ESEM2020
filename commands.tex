


% Usual suspects

\newcommand*{\ie}{i.e.,\@\xspace}
\newcommand*{\eg}{e.g.,\@\xspace}
\newcommand*{\cf}{cf.\@\xspace}
\newcommand*{\RM}{README\@\xspace}
\newcommand*{\GH}{GitHub\@\xspace}
\newcommand*{\CT}{TopFilter\@\xspace}
\newcommand*{\TF}{TopFilter\@\xspace}
%\newcommand*{\CT}{CFTop\@\xspace}
\newcommand*{\MNB}{MNB network\@\xspace}
\newcommand*{\CR}{<RECOMMENDER ACRONYM>\@\xspace}
\newcommand\revised[1]{\textcolor{blue}{#1}}
\newcommand{\code}[1]{{\small \texttt{#1}}}
\newcommand*{\LF}{LibFinder\@\xspace}
\newcommand*{\LC}{LibCUP\@\xspace}
\newcommand*{\LR}{LibRec\@\xspace}

\newcommand{\rqfirst}{\textbf{RQ$_1$}: \emph{Which collaborative filtering configuration brings the best performance to \CT?}~}%\emph{How \CT parameter settings impact on the prediction performance?}~} 
%\newcommand{\rqsecond}{\textbf{RQ$_2$}: \emph{How the parameters have been evaluated?}~}
\newcommand{\rqsecond}{\textbf{RQ$_2$}: \emph{Is the entangled approach able to improve the \MNB's overall performance?}~}
\newcommand{\rqthird}{\textbf{RQ$_3$}: \emph{How \CT behave with respect the \MNB in terms of prediction performance?}~}

%\renewcommand{\hl}{}


%%%
\newcommand*\circled[1]{\tikz[baseline=(char.base)]{\color{black} 
		\node[shape=circle,draw=cyan,fill=black!10!white,inner sep=.3pt] (char) {\sffamily{\small{\textbf{#1}}}};}}
	
\def\checkmark{\tikz\fill[scale=0.4](0,.35) -- (.25,0) -- (1,.7) -- (.25,.15) -- cycle;} 

\makeatletter
\newcommand*{\etc}{%
	\@ifnextchar{.}%
	{etc}%
	{etc.\@\xspace}%
}
\makeatother
\newcommand*{\etal}{et~al.\@\xspace}

\newcommand{\nb}[2]{
	\fbox{\bfseries\sffamily\scriptsize#1}
	{\sf\small$\blacktriangleright$\textit{#2}$\blacktriangleleft$}
}

\newcommand\PN[1]{\textcolor{blue}{\nb{Phuong}{#1}}}
\newcommand\CDS[1]{\textcolor{red}{\nb{Claudio}{#1}}}
\newcommand\JDR[1]{\textcolor{orange}{\nb{Juri}{#1}}}
\definecolor{verylightgray}{gray}{0.95}
%Is the approach able to provide consistent recommendations?

