In recent years, the developer community heavily exploits open source 
repositories during their daily activities. \GH has become one of the most
popular platforms that aggregate these projects and support collaborative 
development activities~\cite{7832894}. \GH recently introduced the possibility 
to tag repositories employing 
\emph{topics}\footnote{\url{https://help.github.com/en/github/administering-a-repository/classifying-your-repository-with-topics}}
 intending to foster the popularity and promote information discovery about 
available projects. Topics are terms used to characterize projects and to 
facilitate their discoverability from software developers that are searching 
for existing projects providing them with some reusable features. 

Assigning repositories with wrong topics can compromise their popularity and 
reachability. To deal with such problems, in 2017 \GH introduced a 
mechanism named repo-topix to suggest topics by relying on information 
retrieval techniques \cite{repo-topix}. To improve repo-topix 
and to explore additional recommendation strategies, a recent work (named \MNB
hereafter) deals with the problem by using a Multinomial Na\"ive Bayesian network to 
recommend relevant topics starting from the README file of the repository of 
interest \cite{10.1145/3383219.3383227}. However, such a tool can recommend 
only \emph{featured} topics, \ie a set of topics, which are curated by 
\GH.\footnote{\url{https://github.com/topics}}


In this paper, we propose \TF, a recommender system that extends the 
recommendation capabilities of the \MNB approach previously mentioned to 
non-featured topics by exploiting collaborative filtering, a widely used 
technique in the recommender system domain 
\cite{Schafer:2007:CFR:1768197.1768208}. Given an initial set of topics already 
assigned to the \GH repository of interest, we encode it in a graph-based 
structure to represent the mutual relationships between repositories and 
topics. From this, a repository-topic matrix is created by following the 
typical user-item structure used in existing collaborative filtering 
applications. Then, we compute a similarity function based on featured vectors 
to recommend the most similar topics.

We evaluate the \TF's prediction performances by changing different parameters 
as well as comparing it with the \MNB approach. As the direct comparison is not 
possible due to the approaches' internal construction, we used a well-defined 
set of metrics used in the literature to evaluate both approaches by 
considering different datasets. Furthermore, we also investigated the combined 
adoption of MNB and \TF to investigate the potential benefits of their combined 
use. 

The contributions of this paper are as follows:
\begin{itemize}
	\item By considering \GH topics as a product to recommend, we improve 
	repositories' popularity by suggesting a list of relevant topics;
	\item We assess the quality of the work employing a well-defined set of 
	metrics commonly used in the recommendation system domain \ie success rate, 
	accuracy, and catalog coverage;
	\item Considering a well-founded approach, we improve it by recommending an 
	extended set of topics.  
\end{itemize}

The paper is structured as follows. Section \ref{sec:Background} presents the 
context of this work by means of a motivating example. In Section 
\ref{sec:ProposedApproach}, we present the \TF approach. Its evaluation is  
presented in Section \ref{sec:Evaluation}. Section 
\ref{sec:ExperimentalResults} discusses relevant findings and related works are 
summarized in  Section \ref{sec:RelatedWorks}. Finally, we conclude the paper 
and discuss possible future work in Section \ref{sec:Conclusions}.
