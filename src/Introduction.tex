In recent years, the open source software (OSS) community 
makes a daily usage of open source repositories to 
contribute their work as well as to access to projects 
coming from other developers. \GH is one of the most 
well-known platforms that aggregate these projects and 
render possible the exchange of knowledge among the users.
%
In order to aid information discovery and help 
developers identify projects that can be of their interest, \GH introduced \emph{topics}. They are 
words used to characterize projects, which thus can be 
annotated by means of lists of words that summarize projects' features. Thanks 
to the availability of \emph{topics}, several applications are enabled, 
including the automated cataloguing of GitHub repositories \cite{davidlo1}, 
further than allowing developers to explore projects by type, technology, and 
more.

Assigning the right topics to \GH repositories is a crucial step that, if not 
properly done, can affect in a negative way their discoverability. In 
2017, \GH presented \textit{repo-topix}, a topic suggestion tool essentially 
based on information retrieval techniques \cite{noauthor_topic_nodate}. 
Although the mechanism works well so far and it is fully integrated in \GH, in our 
opinion there is still some room for improvement, e.g., in terms of the variety 
of the suggested topics, novel data analysis techniques, and the investigation of new recommendation strategies.

In this work, we propose to face these issues by exploiting collaborative filtering, a widely spread technique in the recommendation system domain \cite{Schafer:2007:CFR:1768197.1768208}. By considering \GH projects as items to recommend,  we use user-topic matrixes to perform the similarity and suggest relevant topics given an initial list coming from an input project. 

By considering \GH projects as items to recommend,  we use user-topic matrixes to perform the similarity and suggest relevant topics given an initial list coming from an input project. 
The work gives the following contributions:
\begin{itemize}
\item Considering the \GH projects as products, we suggest relevant topics to the project given an initial list of them;
\item We assess the quality of the work employing a well-defined set of metrics commonly used in the recommendation system domain \ie sales diversity, novelty, and accuracy
\end{itemize}

The rest of the work is structured as follows. Section \ref{sec:Background} shows the issues and the potential challenges in the domain. In Section \ref{sec:ProposedApproach}, we present our approach and evaluate it in Section \ref{sec:Evaluation}. We present the results of the assessment in Section \ref{sec:ExperimentalResults} and we discuss the findings. Section \ref{sec:RelatedWorks} summarizes relevant works in the field and we conclude the paper in Section \ref{sec:Conclusions} with possible future works.
