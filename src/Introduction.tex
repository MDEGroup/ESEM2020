In recent years, the open-source software (OSS) community
makes a daily usage of open source repositories to
contribute their work as well as to access to projects
coming from other developers. \GH is one of the most
well-known platforms that aggregate these projects and
render possible the exchange of knowledge among the users.
%
In order to aid information discovery and help
developers identify projects that can be of their interest, \GH introduced \emph{topics}. They are
words used to characterize projects, which thus can be
annotated by means of lists of words that summarize projects' features. Thanks
to the availability of \emph{topics}, several applications are enabled,
including the automated cataloging of GitHub repositories \cite{davidlo1},
further than allowing developers to explore projects by type, technology, and
more.

Assigning the right topics to \GH repositories is a crucial step that, if not
properly done, can affect in a negative way their discoverability. In
2017, \GH presented \textit{repo-topix}, a topic suggestion tool essentially
based on information retrieval techniques \cite{noauthor_topic_nodate}.
Although the mechanism works well so far and it is fully integrated with \GH, in our opinion there is still some room for improvement, e.g., in terms of the variety
of the suggested topics, novel data analysis techniques, and the investigation of new recommendation strategies.

We have already faced this problem in our previous work \cite{MNB} by using a machine learning approach to recommend relevant topics given a README file of a repository. In this initial attempt, we are able to recommend only \emph{featured} topics, a curated list of them provided by Github \cite{}.

In this work, we propose to extend the set of recommended items to non-featured topics by exploiting collaborative filtering, a widely spread technique in the recommendation system domain \cite{Schafer:2007:CFR:1768197.1768208}. Given an initial set of topics coming from a \GH project, we use repository-topic matrixes to suggest relevant topics.
The work gives the following contributions:
\begin{itemize}
\item Considering the \GH projects as products, we suggest relevant topics to the project given an initial list of them;
\item We assess the quality of the work employing a well-defined set of metrics commonly used in the recommendation system domain \ie sales diversity, novelty, and accuracy;
\item We extend our previous work in the domain considering the entire set of topics and use it as a baseline
\end{itemize}

The rest of the work is structured as follows. Section \ref{sec:Background} shows the issues and the potential challenges in the domain. In Section \ref{sec:ProposedApproach}, we present our approach and evaluate it in Section \ref{sec:Evaluation}. We present the results of the assessment in Section \ref{sec:ExperimentalResults} and we discuss the findings. Section \ref{sec:RelatedWorks} summarizes relevant works in the field and we conclude the paper in Section \ref{sec:Conclusions} with possible future works.

