In recent years, the developer community heavily exploits open source repositories during their daily activities. \GH has become one of the most
popular platforms that aggregate these projects and
support the development activity in a collaborative fashion.
%
The platform recently introduced \emph{topics} to foster the popularity and promote information discovery about popular projects. They are a set of terms used to characterize projects by summarizing their features.  
Thus, the topic labeling activity can compromise the popularity and reachability of a project if it is not properly addressed. A recent work already faced this problem by using a machine learning approach to recommend relevant topics given a README file of a repository  \cite{MNB}. However, this tool\footnote{For the sake of presentation, we refer to this work as MNB network throughout the paper} is able to recommend only \emph{featured} topics, a curated list of them provided by Github\footnote{\url{https://github.com/topics}}.

In this work, we extend the set of recommended items to non-featured topics by exploiting collaborative filtering, a widely spread technique in the recommendation system domain \cite{Schafer:2007:CFR:1768197.1768208}. Given an initial set of topics coming from a \GH project, we use repository-topic matrixes to suggest relevant topics.
The work gives the following contributions:
\begin{itemize}
\item Considering the \GH projects as products, we suggest relevant topics to the project given an initial list of them;
\item We assess the quality of the work employing a well-defined set of metrics commonly used in the recommendation system domain \ie success rate and accuracy;
\item Considering a well-founded approach, we improve it by providing an extended set of possible topics  
\end{itemize}

The rest of the work is structured as follows. Section \ref{sec:Background} shows the issues and the potential challenges in the domain. In Section \ref{sec:ProposedApproach}, we present our approach and evaluate it in Section \ref{sec:Evaluation}. We present the results of the assessment in Section \ref{sec:ExperimentalResults} and we discuss the findings. Section \ref{sec:RelatedWorks} summarizes relevant works in the field and we conclude the paper in Section \ref{sec:Conclusions} with possible future works.
