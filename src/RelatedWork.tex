This section discusses relevant work in this domain.




Immediately after \GH platform introduces topics, they present Repo-Topix, an automatic approach to suggest them \cite{ganesan_topic_2017}. Such a tool relies on parsing the \RM files and the textual content of a repository to enable the standard NLP techniques. Then, they filter this initial set of topics by exploiting the TF-IDF scheme and a regression model to exclude "bad'' topics. As the final step, Repo-Topix computes a custom version of Jaccard Distance to discover additional similar topics. A rough evaluation based on the n-gram ROUGE-1 metrics has been conducted by counting the number of overlapping units between the recommended topics and the repository description. Nevertheless, a replication package with the complete dataset and the source code is not available for further investigation

In~\cite{orii_modeling_2013}, the author proposes a collaborative topic regression (CTR) model to excerpt topics from an initial \GH repository. The final aim is to recommend other similar projects given the input one. Given a pair of user-repository, the approach uses a Gaussian model to compute matrix factorization and extract the latent vectors given a pre-computed matrix rating. Additionally, a probabilistic topic modeling is applied to find topics from the repositories by analyzing high frequent terms. The approach is evaluated by conducting five-fold cross-validation on a dataset composed of  120,867 repositories. Such evaluation considers the pairs user-repository that have at least 3 watches. 

Lia et. al \cite{liao_user_2018} propose a user-oriented portrait model to recommend a set of labels for \GH projects. An initial set of labels is obtained by computing the LDA algorithm on the textual elements of a repository \ie issues, commits, and pull requests. Then, the approach exploits a project familiarity technique that relies on the user's behavior considering the different repositories operation. Such a strategy enables the collaborative filtering technique that exploits two kinds of similarity \ie attribute and social similarity. The former takes into account the personal user information such as the company, the geographical information and the time when the account has been created. The latter computes the similarity scores considering the proportion of items contributed by the user. The approach is evaluated by considering 80 different users with an average of 1894 different behaviors for each one. By considering the first two months of activity in 2016 as a test set, the assessment shows that the approach improves the performances in terms of precision, recall, and success rate


A model-based fuzzy C-means for collaborative filtering (MFCCF) has been proposed in \cite{ajoudanian_recommending_2019} with the aim of recommending relevant human resources during the \GH project development. Similarly to our approach, the proposed model encodes relevant information about repositories in a graph structure and excerpt from it the sparsetest sub-graph. This phase is preparatory to enable the fuzzy C-means clustering technique. Using the computed sparse sub-graph as the center of the cluster, the model can handle the sparsity issue that normally arises in the CF domain. Then, MFCCF computes the Pearson Correlation for each pair user-item belonging to a cluster and retrieves the top-N results. The evaluation is performed using the GHTorrent dump to collect the necessary information. Using ten projects as the testing dataset, the results of the MFCCF are compared with the ones chosen by HR company managers. The results demonstrate the effectiveness of the approach with an accuracy of 80\% on average. 