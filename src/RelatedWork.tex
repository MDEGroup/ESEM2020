
%This section discusses relevant work in this domain.
%Recently, \GH integrates the \emph{repo-topix}~\cite{repo-topix} tool with the aim of recommending topics for a \GH repository. Textual metadata (\ie README files and repository's description) are manipulated by NLP techniques to find a set of topics suitable for describing and classifying the repository. 

Immediately after the introduction of topics in the \GH platform, the 
Repo-Topix tool was presented \cite{ganesan_topic_2017}. Such a tool relies on 
parsing the \RM files and the textual content of a given \GH repository to 
suggest topics automatically. As a first step, the tool applies standard NLP 
techniques on the input artifacts. Then, they filter an initial set of topics 
by exploiting the TF-IDF scheme and a regression model to exclude "bad'' 
topics. As the final step, Repo-Topix computes a custom version of Jaccard 
Distance to discover additional similar topics. A rough evaluation based on the 
n-gram ROUGE-1 metrics has been conducted by counting the number of overlapping 
units between the recommended topics and the repository description. 
Nevertheless, a replication package with the complete dataset and the source 
code of the tool is not available. Thus further investigations and comparisons 
are hampered.



%Nevertheless, neither the tool nor the dataset is made available, thus a comparison with the approach is not feasible.


In~\cite{orii2012collaborative}, the author proposes a collaborative topic 
regression (CTR) model to extract topics from a given \GH repository. The final 
aim is to recommend other similar projects given the input one. Given a pair of 
user-repository, the approach uses a Gaussian model to compute matrix 
factorization and extract the latent vectors given a pre-computed matrix 
rating. Additionally, a probabilistic topic modeling is applied to find topics 
from the repositories by analyzing high frequent terms. The approach is 
evaluated by conducting five-fold cross-validation on a dataset composed of  
120,867 repositories. Such evaluation considers the pairs user-repository that 
have at least 3 watches. Differently from \TF, this approach can recommend \GH 
 repositories that are relevant with respect to the topics of the input 
 repository.

Lia et al. \cite{liao_user_2018} propose a user-oriented portrait model to 
recommend a set of \GH projects that can be of interest for a given user. An 
initial set of labels is obtained by computing the LDA algorithm on the textual 
elements of a repository \ie issues, commits, and pull requests. Then, the 
approach exploits a project familiarity technique that relies on the user's 
behaviour, considering the different repositories operation. Such a strategy 
enables the collaborative filtering technique that exploits two kinds of 
similarity \ie attribute and social similarity. The former takes into account 
the personal user information such as the company, the geographical information 
and the time when the account has been created. The latter computes the 
similarity scores considering the proportion of items contributed by the user. 
The approach is evaluated by considering 80 different users with an average of 
1,894 different behaviours for each one. By considering the first two months of 
activity in 2016 as a test set, the assessment shows that the approach improves 
the performances in terms of precision, recall, and success rate of the 
considered baseline. Even though both \TF and the work presented in 
\cite{liao_user_2018} make use of collaborative-filtering techniques, the 
former is designed to recommend topics to be assigned to an input \GH 
repository, whereas the latter recommends \GH repositories that can be of 
interest for a given user.


A model-based fuzzy C-means for collaborative filtering (MFCCF) has been 
proposed in \cite{ajoudanian_recommending_2019} to recommend relevant human 
resources during the \GH project development. Similarly to our approach, the 
proposed model encodes relevant information about repositories in a graph 
structure and extract from it the sparse test sub-graph. This phase is 
preparatory to enable the fuzzy C-means clustering technique. Using the 
computed sparse sub-graph as the centre of the cluster, the model can handle 
the sparsity issue that generally arises in the CF domain. Then, MFCCF computes 
the Pearson Correlation for each pair user-item belonging to a cluster and 
retrieves the top-N results. The evaluation is performed using the GHTorrent 
dump to collect the necessary information. Using ten projects as the testing 
dataset, the results of MFCCF are compared with the ones chosen by the human 
resource department of the considered company. The results demonstrate the 
effectiveness of the approach with an accuracy of 80\% on average. 

The REPERSP tool \cite{xu_repersp_2017} recommends \GH projects by exploiting 
users' behaviour. As the first step, the tool computes the similarities between 
projects using the TF-IDF weighting scheme to obtain the content similarity 
matrix. Additionally, REPERSP captures the developer's behaviour by considering 
her activity on \GH \ie create, star, and fork actions over projects. A 
different value is assigned for each type of action to create a user-project 
matrix. Finally, the tool combines the two similarity matrixes to deliver the 
recommended projects. To assess the quality of the work, REPERSP is compared 
with the traditional collaborative filtering techniques \ie user-based and 
item-based. The study is conducted over two groups with different users, 
projects, and purposes. The results show that the proposed tool outperforms the 
considered baseline in terms of accuracy, precision, and recall. 

Besides \GH projects, tags and topics are successfully used in different 
contexts \ie in social networks. Purushotham et al. 
\cite{purushotham_collaborative_nodate} propose a hierarchical Bayesian model 
that relies on a topic model to provide final users with relevant items. By 
combining  LDA and matrix factorization techniques, the proposed model is able 
to reduce the sparsity problem that typically occurs when the collaborative 
filtering is employed. After this preprocessing phase, the hierarchical 
Bayesian model is tuned with several parameters to maximize the prediction 
performances. They are evaluated by using two large real-world datasets 
tailored for music and bookmark recommendations. The results show that the 
proposed model overcomes the classical CTR model in terms of recall. 

%Bog\'ardi-M\'esz\"oly \etal \cite{bogardi-meszoly_tag_2013} propose two 
%different recommender systems based on vocabulary have been proposed \ie tags 
%and topics 
%recommenders. The first filters the initial thesaurus terms using algorithms 
%to 
%correct spelling and clerical errors. Then, the filtered tags are grouped into 
%distinct classes to highlight the popular ones. The topics recommender 
%improves 
%the former model by constructing a graph from the tags. The following steps 
%involve the identification of relevant topics and their graphical 
%visualization 
%in the so-called topic clouds. The soundness of the approach is assessed by 
%comparing the obtained outcomes with the real-world thesis portal 