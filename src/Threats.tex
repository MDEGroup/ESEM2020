\CDS{Check if this section makes sense}
This section discusses the threats that may affect the results of the evaluation. We also list the countermeasures taken to minimize these issues.
\emph{Internal validity} concerns the rationale behind the selection of the \GH repository used in the assessment. As stated in the related section, we download randomly repositories by imposing the quality filter on the stars. Nevertheless, some repositories can be tagged with topics that can affect the quality of the graph computed in the data extraction phase. To be concrete, a user can label its repository using terms that are not enough descriptive \ie using infrequent or duplicated terms in the topic list. To deal with this issue, we apply the topic filter as stated in Section \ref{sec:} to reduce the possible noise during the graph building.
Threats to \emph{external validity} concerns the choice of the \MNB as the baseline in the conducted experiment. First of all, the replication package to run the tool is available and it allows a more comprehensive evaluation rather than using only metrics. Additionally, the \MNB faced already the problem of recommending \GH topics but it covers only featured topics. As we claimed before, the two approaches are strongly different from the construction point of view including the recommendation engine and data extraction components. To make the comparison as fair as possible, we run the \MNB on the same datasets by adapting the overall structure for the ten folder validation.